\documentclass[
	12pt,				
	openright,		
	twoside,	
	a4paper,
	english,	
	brazil	
	]{abntex2}
\usepackage{lmodern}		
\usepackage[T1]{fontenc}	
\usepackage[utf8]{inputenc}
\usepackage{lastpage}	
\usepackage{indentfirst}
\usepackage{color}	
\usepackage{graphicx}
\usepackage{microtype}
\graphicspath{{./imagens/}}
\usepackage{lipsum}			
\usepackage[brazilian,hyperpageref]{backref}	 
\usepackage[alf]{abntex2cite}	
\usepackage{listings}
\usepackage{multirow}
\usepackage{array}
\newcolumntype{M}[1]{>{\centering\arraybackslash}m{#1}}
\renewcommand{\backrefpagesname}{Citado na(s) página(s):~}
\renewcommand{\backref}{}
\renewcommand*{\backrefalt}[4]{
	\ifcase #1 %
		Nenhuma citação no texto.%
	\or
		Citado na página #2.%
	\else
		Citado #1 vezes nas páginas #2.%
	\fi}%
        \titulo{Avaliando Sistemas de Detecção de Intrusão em uma Rede Acadêmica}
\autor{Glenon Mateus Barbosa Araújo}
\local{Brasil}
\data{\the\year}
\orientador{Dr. Roberto Samarone dos Santos Araújo}
%\coorientador{}
\instituicao{
  Universidade Federal do Pará -- UFPA
  \par
  Faculdade de Computação
  \par
  Bacharelado em Ciência da Computação}
\tipotrabalho{Trabalho de Conclusão de Curso}
\preambulo{Trabalho de Conclusão de Curso submetida a graduação em Ciência da Computação da UFPA}
\definecolor{blue}{RGB}{41,5,195}
\makeatletter
\hypersetup{
     	%pagebackref=true,
		pdftitle={\@title}, 
		pdfauthor={\@author},
    	pdfsubject={\imprimirpreambulo},
	    pdfcreator={LaTeX with abnTeX2},
		pdfkeywords={abnt}{latex}{abntex}{abntex2}{trabalho acadêmico}, 
		colorlinks=true,       	
    	linkcolor=blue,          
    	citecolor=blue,        	
    	filecolor=magenta,     
		urlcolor=blue,
		bookmarksdepth=4
}
\makeatother
\setlength{\parindent}{1.3cm}
\setlength{\parskip}{0.2cm} 
\makeindex
\begin{document}
\selectlanguage{brazil}
\frenchspacing 
\imprimircapa
\imprimirfolhaderosto*

\begin{fichacatalografica}
	\sffamily
	\vspace*{\fill}
	\begin{center}
	\fbox{\begin{minipage}[c][8cm]{13.5cm}	
	\small
	\imprimirautor

	\hspace{0.5cm} \imprimirtitulo  / \imprimirautor. --
	\imprimirlocal, \imprimirdata-
	
        \hspace{0.5cm} \pageref{LastPage} p. : il. (algumas color.) ; 30 cm.\\
	
        \hspace{0.5cm} \imprimirorientadorRotulo~\imprimirorientador\\
	
        \hspace{0.5cm}
        \parbox[t]{\textwidth}{\imprimirtipotrabalho~--~\imprimirinstituicao,
	\imprimirdata.}\\
	
        \hspace{0.5cm}
		1. Suricata.
		2. Snort.
		3. IDPS.
		I. Orientador.
		II. Universidade Federal do Pará.
		III. Faculdade de Computação.
		IV. Análise de IDPSs
	\end{minipage}}
	\end{center}
\end{fichacatalografica}

% \begin{errata}
%Elemento opcional da \citeonline[4.2.1.2]{NBR14724:2011}. Exemplo:
%\vspace{\onelineskip}
%FERRIGNO, C. R. A. \textbf{Tratamento de neoplasias ósseas apendiculares com
%reimplantação de enxerto ósseo autólogo autoclavado associado ao plasma
%rico em plaquetas}: estudo crítico na cirurgia de preservação de membro em
%cães. 2011. 128 f. Tese (Livre-Docência) - Faculdade de Medicina Veterinária e
%Zootecnia, Universidade de São Paulo, São Paulo, 2011.
%\begin{table}[htb]
%\center
%\footnotesize
%\begin{tabular}{|p{1.4cm}|p{1cm}|p{3cm}|p{3cm}|}
%  \hline
%   \textbf{Folha} & \textbf{Linha}  & \textbf{Onde se lê}  & \textbf{Leia-se}  \\
%    \hline
%    1 & 10 & auto-conclavo & autoconclavo\\
%   \hline
%\end{tabular}
%\end{table}
%\end{errata}

\begin{folhadeaprovacao}
  \begin{center}
    {\ABNTEXchapterfont\large\imprimirautor}
    \vspace*{\fill}\vspace*{\fill}
    \begin{center}
      \ABNTEXchapterfont\bfseries\Large\imprimirtitulo
    \end{center}
    \vspace*{\fill}
    \hspace{.45\textwidth}
    \begin{minipage}{.5\textwidth}
        \imprimirpreambulo
    \end{minipage}%
    \vspace*{\fill}
   \end{center}
   Trabalho aprovado. \imprimirlocal, 12 de julho de 2018:
   \assinatura{\textbf{\imprimirorientador} \\ Orientador} 
   %\assinatura{\textbf{Professor} \\ Convidado 1}
   %\assinatura{\textbf{Professor} \\ Convidado 2}
   %\assinatura{\textbf{Professor} \\ Convidado 3}
   %\assinatura{\textbf{Professor} \\ Convidado 4}
   \begin{center}
    \vspace*{0.5cm}
    {\large\imprimirlocal}
    \par
    {\large\imprimirdata}
    \vspace*{1cm}
  \end{center}
\end{folhadeaprovacao}

%\begin{dedicatoria}
%   \vspace*{\fill}
%   \centering
%   \noindent
%   \textit{•} \vspace*{\fill}
%\end{dedicatoria}
%\begin{agradecimentos}
%\end{agradecimentos}
%\begin{epigrafe}
%    \vspace*{\fill}
%	\begin{flushright}
%		\textit{•}
%	\end{flushright}
%\end{epigrafe}

\setlength{\absparsep}{18pt} 

\begin{resumo}
 
    A todo momento novos ataques ou mesmo variações de ataques existentes surgem e são lançados a  redes. Somente a utilização de \textit{firewall} é insuficiente pois o atacante pode utilizar portas, na qual rodam serviços fornecidos pela empresa/instituição, para realizar ataques. Diante desse cenário, é indispensável, para um administrador de rede, o uso de ferramentas auxiliares como um Sistema de Detecção e Prevenção de Intrusão (IDPS), que alertam ou até mesmo bloqueiam ataques. Esse trabalho tem o objetivo de avaliar os Sistemas de Detecção de Intrusão \textit{open source} mais populares, Snort e Suricata. O Snort é um sistema de detecção lançado em 1998 por Martin Roesch foi uma das primeiras em seu segmento a realizar análise de tráfego em tempo real e registro dos pacotes de forma leve, utilizando recursos mínimos de processamento. Já o Suricata, foi lançado em 2010, e tem como principal característica utilização da tecnologia \textit{multithreading}, tirando maior proveito dos processadores, visando melhorar o desempenho. A avaliação foi realizada comparando o desempenho e as detecções de intrusões as quais foram colocadas em funcionamento em uma rede de produção real, verificando suas vantagens e desvantagens. Nos testes realizados, foram utilizados ferramentas auxiliares para simular ataques a uma máquina alvo instalada para essa finalidade. 

\textbf{Palavras-chave}: Segurança da Informação, Suricata, Snort, Sistema de Detecção de Intrusão, Sistema de Prevenção de Intrusão, IDS, IPS.
\end{resumo}
\begin{resumo}[Abstract]
 \begin{otherlanguage*}{english}
   \vspace{\onelineskip}
   \noindent 
   At any moment new attacks or even variations of existing attacks arise and are launched only on networks. Only the use of firewall is insufficient because the attacker can use ports, in which services provided by the company / institution run, to execute attacks. Given this scenario, it is essential for a network administrator to use tools such as a Intrusion Detection and Prevention System that alert or even block attacks. This work aims to evaluate the most popular open source Intrusion Detection Systems, Snort and Suricata. Snort is a detection system launched in 1998 by Martin Roesch and was one of the first in its segment to perform real-time traffic analysis and package logging in a lightweight manner using minimal processing capabilities. Suricata was launched in 2010 and its main feature is the use of multithreading technology, taking advantage of processors to improve performance. The evaluation was performed comparing the performance and intrusion detections that were put into operation in an actual production network, verifying their advantages and disadvantages. In the tests performed, auxiliary tools were used to simulate attacks on a target machine installed for this purpose.

   \textbf{Keywords}: Security Information, Suricata, Snort, Intrusion Detection System, Intrusion Prevention System, IDS, IPS.
 \end{otherlanguage*}
\end{resumo}
\pdfbookmark[0]{\listfigurename}{lof}
\listoffigures*
\cleardoublepage
\pdfbookmark[0]{\listtablename}{lot}
\listoftables*
\cleardoublepage
\begin{siglas}
  \item[IDS] \textit{Intrusion Detection System}
  \item[IPS] \textit{Intrusion Prevention System}
  \item[SDI] \textit{Sistema de Detecção de Intrusão}
  \item[SPI] \textit{Sistema de Prevenção de Intrusão}
  \item[IDPS] \textit{Intrusion Detection and Prevention System}
  \item[HIDS] \textit{Host Based Intrusion Detection Systems}
  \item[NIDS] \textit{Network Based Intrusion Detection Systems}
  \item[MB] \textit{Megabytes}
  \item[GB] \textit{Gigabytes}
  \item[SO] \textit{Sistema Operacional}
  \item[JSON] \textit{JavaScript Object Notation}
  \item[CAIS] \textit{Centro de Atendimento a Incidentes de Segurança}
  \item[DoS] \textit{Denial of Services}
  \item[DDoS] \textit{Distributed Denial of Services}
  \item[URL] \textit{Uniform Resource Locator}
  \item[LAN] \textit{Local Area Network}
  \item[WAN] \textit{Wide Area Network}
  \item[DMZ] \textit{Demilitarized Zone}
  \item[SPF] \textit{Stateful Packet Filter}
  \item[SQL] \textit{Structured Query Language}
  \item[XSS] \textit{Cross-site scripting}
  \item[OISF] \textit{Open Information Security Foundation}
  \item[ET] \textit{Emerging Threats}
  \item[NVT] \textit{Network Vulnerability Tests}
  \item[OMP] \textit{OpenVAS Management Protocol}
  \item[OAP] \textit{OpenVAS Assistant Protocol}
  \item[GSD] \textit{Greenbone Security Desktop}
  \item[GSA] \textit{Greenbone Security Assistant}
  \item[VM] \textit{Virtual Machine}
\end{siglas}
\pdfbookmark[0]{\contentsname}{toc}
\tableofcontents*
\cleardoublepage
\textual
\chapter{Introdução} \label{ch:introdução}
%- Introdução: Texto introdutório do trabalho, motivação, objetivos,
%metodologia, trabalhos relacionados.
%- Em cada capitulo adicionar um texto introdutório
%- Adicionar o seguinte texto:
%Este capítulo esta organizado da seguinte forma: A próxima seção
%apresenta .... A Seção XY apresenta...
Este trabalho apresenta um comparativo entre sistemas de detecção e prevenção de intrusão (Intrusion Detection and Prevention System - IDPS) de código aberto mais populares na comunidade de segurança da informação. Essas ferramentas permitem monitorar sistemas computacionais, alertando os administradores sobre possíveis ameaças.

Neste capítulo, apresentam-se, na \autoref{sec:motivação}, as motivações deste trabalho, evidenciando a importância do IDPS em um ambiente corporativo real. Em seguida, no \autoref{sec:objetivos}, os objetivos do trabalho, na \autoref{sec:metodologia}, uma descrição das metodologias utilizadas, na \autoref{sec:trabalhos-relacionados} os trabalhos relacionados, e por fim, na \autoref{sec:organização-do-trabalho}, a organização do trabalho.

\section{Motivação} \label{sec:motivação} 

A Internet é um conjunto de redes físicas heterogênea (uma variedade de dispositivos conectados, \textit{smartphones}, \textit{desktops}, \textit{notebooks}, servidores, \textit{switches}, roteadores, entre outros) funcionando como uma rede lógica única de alcance mundial. O grande e contínuo crescimento da Internet gerou um aumento da sua complexidade, que a expõe a diversas vulnerabilidades. 

A todo momento, novos ataques ou mesmo variações de ataques já existentes surgem e são lançados a várias redes indiscriminadamente em busca de vulnerabilidades. Em 2015, foram reportados ao Centro de Estudos, Respostas e Tratamento de Incidentes de Segurança no Brasil (CERT.br) cerca de 722.205 incidentes, em 2016, esse número diminuiu, chegando a 647.112 \cite{estatistica:cert.br}. Apesar da diminuição, esse número é considerado grande, presumindo-se que há muitos incidentes que não são reportados e/ou identificados.

As empresas, de qualquer segmento e tamanho, devem ter o trabalho de manter os ativos seguros e isso vai além da utilização de anti-vírus nos computadores. Ter uma política de atualização de \textit{software} em estações de trabalho e servidores, aliados com boas práticas nas configurações de serviços, dificultam a exploração de vulnerabilidades. 

A utilização de \textit{firewall}, não pode ser encarado com uma solução definitiva, passando uma falsa sensação de segurança, pois muitas portas legítimas podem estar vulneráveis, como acontece, por exemplo, com a porta 80 que hospeda \textit{websites} vulneráveis \cite{analisenessus:cleriston}.

Diante desse cenário, torna-se cada vez mais importante para o administrador de rede e/ou segurança da informação o uso de ferramentas de IDPS, permitindo identificar e tratar de forma automatizada os incidentes de segurança.

Para implementa tais ferramentas em uma rede, deve-se levar em consideração a flexibilidade e a administração simplificada para não resultar que a empresa/instituição fique na dependência de um único fabricante ou fornecedor da solução. Além disso, a ferramenta deve ter um bom desempenho e eficiência para não passar a falsa sensação de proteção ou degradar o desempenho da rede.

\section{Objetivos} \label{sec:objetivos}

Diante do apresentado, este trabalho tem como objetivo geral, analisar e apresentar um comparativo entre as soluções de código aberto de IDPS mais conhecidas: Snort e Suricata. Como desdobramento de tal objetivo, os seguintes objetivos secundários foram definidos:

\begin{alineas}
\item Apresentar conceitos sobre segurança da informação e sua importância em um ambiente corporativo;
\item Descrever problemas relacionados a ataques envolvendo redes de computadores;
\item Descrever as ferramentas, compreendendo requisitos, características, modos de atuação e funcionalidades;
\item Descrever o ambiente experimental;
\item Realizar experimentos e coletar dados para validar o funcionamento e a eficiência das ferramentas;
\end{alineas}

\section{Metodologia} \label{sec:metodologia}

Esse trabalho se configura numa pesquisa qualitativa. Utilizou-se uma máquina com \textit{hardware} robusto, configurada com 130G de memória RAM, com um processador possuindo 40 núcleos e duas interfaces de rede, uma para gerência e outra configurada em modo '\textit{promisc}' com espelhamento do roteador. Nela instalou-se o SO XenServer versão 7, da Citrix, SO voltado para virtualização com um bom desempenho num ambiente de produção real. 

No \textit{host}, instalou-se duas máquinas virtuais, uma para cada ferramenta de IDPS em teste, Snort e Suricata, nas suas versões mais recentes, 2.9.8.3 e 4.0.0, respectivamente. A mesma quantidade de recurso de \textit{hardware} foi alocada para as máquinas, criando assim um ambiente igual para ambas as ferramentas. Além disso, após configuradas, usou-se a mesma base de assinaturas aberta da \textit{Emerging Threats} (ET), que possui frequentes atualizações.

As métricas para comparação avaliadas são a quantidade de recurso usado pela ferramenta para análise do tráfego em um período de tempo determinado, quantidade de falsos positivos e negativos e a taxa de detecção. Para avaliar os recursos de \textit{hardware}, usou-se dois recursos. Primeiro, foi instalado nas máquinas virtuais um \textit{daemon} collectd, a segunda forma, foi instalar um servidor de monitoramento Zabbix. 

Para facilitar a avaliação dos alertas gerados pelas ferramentas e determinar os falsos positivos e negativos, optou-se por centralizar o \textit{logs} em um servidor. Para tal, usou-se uma infraestrutura que reúne três serviços, descrito na \autoref{sec:infraestrutura}.

\section{Trabalhos Relacionados} \label{sec:trabalhos-relacionados}

O uso de ferramentas IDS \textit{opensource} Snort e/ou Suricata, importantes na área de segurança, já foi abordado e apresentou alguns resultados satisfatórios em pesquisas anteriores, por exemplo, nas pesquisas de Nagahama \textit{et al.} (2013), Martín \textit{et al.} (2014) e Cléber \textit{et al} (2014).

No trabalho do Nagahama \textit{et al.} (2013), é usado redes definidas por \textit{softwares}, que desacopla os planos de controle e de dados, permitindo adaptar o funcionamento da rede de acordo com a necessidade de cada um. O \textit{software} utilizado na pesquisa foi o OpenFLOW. Tal uso, tem como objetivo mitigar a falta de integração do IDS com os equipamentos de rede como {switches} e roteadores, o que limita a atuação destas ferramentas. 

Já Martín \textit{et al.} (2014), utiliza-se do BroFlow que possui uma série de vantagem, como, detecção de intrusão através de algoritmos simples, modular e flexível, reação imediata a um ataque descantando pacotes dos atacantes os mais próximo da origem. Dentre os resultados obtidos, destacam-se que a ferramenta conseguiu garantir o encaminhamento de pacotes legítimos na rede na taxa máxima do enlace e reduziu, em até dez vezes, o atraso na rede provocado pelo ataque.

No trabalho de Cléber \textit{et al} (2014), foi feito uma comparação de desempenho das ferramentas de IDS (Snort e Suricata) porém usou-se dados sintéticos fornecidos pela \textit{Defense Advanced Research Projects Agency} (DARPA). Ao final, listou-se as vantagens e desvantagens existentes de cada ferramenta. No trabalho proposto, no entanto, serão usados dados mais próximo de um ambiente de produção.

\section{Organização do Trabalho} \label{sec:organização-do-trabalho}

Além deste capítulo introdutório, esse trabalho está divido da seguinte forma:

No \autoref{ch:segurança}, são definidos conceitos sobre redes de computadores e segurança da informação, também são descritos os ataques comuns e ferramentas utilizadas para validar e avaliar as soluções de IDPS.

Em seguida, no \autoref{ch:idps}, a definição IDPS, os tipos existentes, as funcionalidades e descrição das ferramentas avaliadas: Snort e Suricata.

O \autoref{ch:cenário-real} detalhará o cenário real e a infraestrutura utilizada para os realização dos testes das ferramentas, os testes realizados e o resultados obtidos.

Por fim, no \autoref{ch:considerações}, as considerações finais e trabalhos futuros.

\chapter{Segurança em Redes de Computadores} \label{ch:segurança}
\section{Definições} \label{sec:definições}

Para entendemos melhor o que é segurança da informação, precisamos conceituar alguns termos que serão detalhados abaixo \cite{esr-gestao}:

\begin{alineas}
 \item \textbf{Incidente de segurança}: qualquer evento oposto a segurança; por exemplo, ataques de negação de serviços (Denial of Service - DoS), roubo de informações, vazamento e obtenção de acesso não autorizado a informações;
 \item \textbf{Ativo}: qualquer coisa que tenha valor para a organização e para seus negócios. Alguns exemplo: banco de dados, softwares, equipamentos (computadores e notebooks), servidores, elementos de redes (roteadores, switches, entre outros), pessoas, processos e serviços;
 \item \textbf{Ameaça}: qualquer evento que explore vulnerabilidades. Causa potencial de um incidente indesejado, que pode resultar em dano para um sistema ou organização;
 \item \textbf{Vulnerabilidade}: qualquer fraqueza que possa ser explorada e comprometer a segurança de sistemas ou informações. Fragilidade de um ativo ou grupo de ativos que pode ser explorada por uma ou mais ameaças. Vulnerabilidades são falhas que permitem o surgimento de deficiências na segurança geral do computador ou da rede. Configurações incorretas no computador ou na segurança também permitem a criação de vulnerabilidades. A partir dessa falha, as ameaças exploram as vulnerabilidades, que, quando concretizadas, resultam em danos para o computador, para a organização ou para os dados pessoais;
 \item \textbf{Risco}: probabilidade de uma ameaça se concretizar;
 \item \textbf{Ataque}: qualquer ação que comprometa a segurança de uma organização;
 \item \textbf{Impacto}: consequência de um evento.
\end{alineas}

Segurança da Informação proteção das informações, sistemas, recursos e demais ativos contra desastres, erros (intencionais ou não) e manipulação não autorizada, objetivando a redução da probabilidade e do impacto de incidentes de segurança.

Segundo a norma ISO/IEC 27002 \cite{isoiec27002}, segurança da informação é a preservação da confidencialidade, da integridade e da disponibilidade da informação; adicionalmente, outras propriedades, tais como autenticidade, responsabilidade, não repúdio e confiabilidade, podem também estar envolvidas.

Dentre vários conhecimentos que um profissional de segurança deve possuir, o conceito mais básico e considerado o pilar de toda a área de segurança corresponde à sigla CID (Confidencialidade, Integridade e Disponibilidade), de modo que um incidente de segurança é caracterizado quando uma dessas áreas é afetada \cite{seg-redes-sistemas}. Abaixo será detalhado cada item.

\begin{alineas}
 \item \textbf{Confidencialidade}: termo ligado à privacidade de um ativo ou recurso, que deve ser acessível somente por pessoas ou grupos autorizados;
 \item \textbf{Integridade}: possui duas definições, a primeira está relacionada com o fato da informação ter valor correto, a segunda, está ligada à inviolabilidade da informação;
 \item \textbf{Disponibilidade}: está relacionada ao acesso à informação, que deve está disponível quando necessária.
\end{alineas}

Dois dos termos citados são fáceis de ser monitorados pois é perceptível para o usuário: a integridade (identificar se uma informação foi alterada) e a disponibilidade (tentando acessar um serviço e verificando se o mesmo está respondendo adequadamente). No entanto, só é possível identificar se houve quebra da confiabilidade através de auditorias, analisando os registros de acesso (se houver), tornando a identificação complicada e em muitos casos impossível \cite{seg-redes-sistemas}.

Além dos conceito listados, a literatura moderna considera mais alguns conceitos auxiliares, temos:

\begin{alineas}
 \item \textbf{Autenticidade}: garantia que uma informação, produto ou documento foi elaborado ou distribuído pelo autor a quem se atribui;
 \item \textbf{Legalidade}: garantia de que ações sejam realizadas em conformidade com os preceitos legais vigentes e que seus produtos tenham validade jurídica;
 \item \textbf{Não repúdio}: conceito bastante utilizado em certificação digital, onde o emissor de uma mensagem não pode negar que a enviou;
 \item \textbf{Privacidade}: habilidade de uma pessoa controlar a exposição e a disponibilidade de informações acerca de si.
\end{alineas}

\section{Cenário Geral} \label{sec:geral}
%apresentar o cenário geral de uma rede conectada a Internet
\section{Pontos de Vulnerabilidade} \label{sec:vulnerabilidade}
%Ex.: Roteador, firewall, clientes e suas aplicações,
\section{Ataques Comuns à Redes de Computadores} \label{sec:ataques}

Nessa seção será descritos os ataques mais comuns à redes e serviços de organizações privadas e públicas, financeiras ou acadêmicas. Para licitar os ataques dessa seção, levou-se em consideração as estatísticas divulgada pelo CERT.br (\autoref{fig:cert}).

O CERT.br é o grupo de resposta a incidentes de segurança para a internet brasileira, mantido Comitê Gestor da Internet no Brasil. Atua na notificação e tratamento de incidentes de segurança dando apoio no processo de resposta. Além disso, faz um trabalho de conscientização e treinamento sobre problemas de segurança no Brasil. 

\begin{figure}[htb]
 \centering
 \caption{Estatísticas de ataques reportadas ao CERT.br}
 \includegraphics[scale=.5]{incidentes-reportados.png}
 \legend{Fonte: \cite{tipos-ataques:certs.br}}
 \label{fig:cert}
\end{figure}

O CAIS é responsável por zela pela segurança da rede Ipê (infraestrutura de rede dedicada à comunidade brasileira de ensino superior), detectando, resolvendo e prevenindo incidentes de segurança. Além disso, tem o papel de orientar (através de publicações de cartilhas) e disseminar boas práticas de segurança da informação, educando e conscientizando usuários de todos os níveis sobre os principais riscos em segurança da informação \cite{cais}.

\subsection{Varredura de Redes} \label{sec:varredura}


\subsection{Exploração de Vulnerabilidades} \label{sec:exploração}

Conforme tratado na \autoref{sec:definições}, uma vulnerabilidade é a fraqueza em sistemas de informação, procedimentos de segurança do sistema e controles internos, ou aplicação que pode ser explorada tendo como origem uma ameaça. 

Uma maneira de reunir informações sobre vulnerabilidade de um alvo ou uma rede inteira é a utilização de \textit{scanners} de vulnerabilidades automatizados. Existem vários tipos de \textit{scanners} para uma variedade de situações, uns específicos para aplicações web, outros para servidores e serviços, como banco de dados. Normalmente, essas ferramentas funcionam em três estágios \cite{}: 

\begin{alineas}
 \item \textbf{Configuração}: aqui será definido o endereço IP do alvo ou a URL (Uniform Resource Locator) da aplicação Web e demais parâmetros, como, por exemplo, utilização de \textit{proxy}.
 \item \textbf{Rastreamento}: esse estágio é especifico de \textit{scanners} de vulnerabilidade de aplicações web, nele, o \textit{scanner} chama a primeira página web e então examina seu código procurando \textit{links}. Cada \textit{link} encontrado é registrado e este procedimento é repetido várias vezes até que \textit{links} e páginas não sejam mais encontrados.
 \item \textbf{Exploração}: vários testes são executados e as requisições e respostas são armazenadas e analisadas. Ao final, os resultados são exibidos ao usuário e podem ser salvos para uma análise posterior. 
\end{alineas}

Os \textit{Scanners} de vulnerabilidades automatizados contêm, e atualizam regularmente, enormes bancos de dados de assinaturas de vulnerabilidades conhecidas para basicamente tudo o que está recebendo de informações em uma porta de rede, inclusive sistemas operacionais, serviços e aplicativos web. Eles podem até detectar vulnerabilidades no software do lado do cliente mediante credenciais suficientes, uma estratégia que pode ser útil em estágios posteriores do ataque, quando o invasor pode estar interessado em expandir ainda mais sua base de operações, comprometendo contas adicionais de usuário privilegiado \cite{hackers:stuart-joel}.

Diante da lista de vulnerabilidades encontradas com os \textit{scanners}, próximo passo seria usar um \textit{exploit} adequado. Os \textit{exploit} são pequenos utilitários usados para explorar vulnerabilidades específicas, podendo serem usados de forma \textit{"stand alone"}, ou seja, serem usados diretamente, ou podem ser incorporados à \textit{malwares} \cite{exploit:cassio}.

\subsection{Força Bruta} \label{sec:forçabruta}

Na segurança da informação, a autenticação é umas das áreas-chaves onde há a distinção de usuários autorizados de outros não-autorizados, tendo como principal vantagem ser de fácil implementação, não requerendo equipamentos, como leitores biométricos \cite{denise-lilian}.

Na literatura sobre segurança da informação, o fator humano é considerado o elo mais fraco. Muitos usuários, por conveniência, criam senhas de acesso fáceis e, em muitos casos, única para acessar diversos sistemas. Nesse ponto que \textit{hackers} iram atuar para ter acesso não-autorizado ao sistema. 

Existem três métodos mais usados por programas de quebra de senha: ataques de dicionário (ou lista de palavras), ataques híbridos e ataques de força-bruta. Nos ataques por dicionários, utilizam-se listas de palavras comuns: nomes próprios, marcas conhecidas, gírias, nomes de canções, entre outros, tais elementos conseguidos por engenharia social \cite{univhacker}. 

 Um ataque de força bruta consiste em gerar todas as permutações e combinações possíveis de senha, criptografar cada uma e comparar a senha gerada com a senha criptografada original até encontrar uma que seja igual \cite{md5crack2012}. 

 Esse tipo de ataque é facilmente detectável pois, além de gerar uma alta carga no servidor, gera uma grande quantidade de registros de logs. No entanto, caso a pessoa má intencionada, de alguma outra forma, tenha acesso ao arquivo de \textit{hash} ou a tabela de usuário de um banco de dados, com as senha criptografadas do sistema, ela pode usar o ataque de força bruta no arquivo em qualquer máquina, assim, impossibilitando a detecção do ataque.

 Muitos sistemas já possuem formas de contornar esse tipo de ataque, por exemplo, bloqueio de usuário ao errar a palavra-chave por uma certa quantidade de vezes. Outra forma, é colocar um tempo de expiração da senha, por exemplo, a senha deve ser trocada a cada trinta dias por uma diferente e nunca usada anteriormente, dessa maneira, inviabilizando a quebra de senha por força bruta. 

 \subsection{Desfiguração de páginas} \label{sec:desfiguração}

 A desfiguração de páginas, \textit{defacement} ou pichação ocorre quando o conteúdo da página \textit{web} de um site é alterado. O atacante (\textit{defacer}) consegue fazer alterações em páginas explorando vulnerabilidade nas aplicações \textit{web} que permite injeção de \textit{script} malicioso ou através de furto de senha de acesso à interface \textit{web} usadas para administração remota \cite{certs-ataques}.

 Nos serviços \textit{web}, como por exemplo, apache2 existe um usuário especial, comumente chamado de www-data ou algo semelhante. O servidor \textit{web}, na maioria das vezes, precisa apenas de permissões de leitura nos arquivos porém muitos gerentes de sistemas cujo a conscientização sobre segurança é insuficiente, designa permissões errôneas (escrita ou alteração), e caso haja um comprometimento, através, por exemplo, de injeção de código remoto PHP, do servidor, o atacante poderá alterar a maioria dos arquivos. A ocorrência amplamente disseminada de ataques de desfiguração de páginas Web é uma consequência direta dessa prática \cite{seguranca:william-lawrie}.

 Em 2010, o site Zone-h registrou mais de 1,4 milhões de páginas desfiguradas, muitas delas associadas a ataques \textit{Cross-site scripting} (XSS). Os \textit{defacements} são considerados ataques passivos pois é gerado apenas uma mensagem na tela \cite{angelo-xss}.

 \subsection{Negação de Serviços} \label{sec:negação}
 
Um ataque de negação de serviço (\textit{Denial of Service} - DoS) tem como principal objetivo deixar um serviço (servidor web, banco de dados) ou recurso (memória, processador)  indisponível, impossibilitando que usuário legítimos tenham acesso a esses recursos. Para tal, o atacante gera diversas requisições inúteis para o servidor, consumindo seus recursos até que o serviço não esteja mais disponível ou degradando a qualidade do serviço \cite{cryptsec}.

Esse tipo de ataque pode gerar grandes prejuízos financeiros para as empresas, principalmente \textit{e-commence}, pois enquanto o sistema está fora ou com uma resposta lenta, as transações financeiras são prejudicadas. Com isso, cria-se também, uma insatisfação pelo usuário do serviço prestado pela empresa.

Existe um forma mais sofisticada de ataque de Negação de Serviço chamada Negação de Serviço Distribuído (\textit{Distributed Denial of Services} - DDoS), enquanto o DoS básico as requisições partem de apenas uma fonte (Figura \ref{dos}), no DDoS o atacante tem acesso a um grande número de computadores (\textit{zombies}) explorando suas vulnerabilidades criando o que chamamos de \textit{botnet} (Figura \ref{ddos}). Com isso, basta o atacante indicar as coordenadas de um ou mais alvos para o ataque \cite{zargarjoshitipper}. 

 \subsection{Malwares} \label{sec:malwares}

 Os \textit{malwares}, também conhecidos como \textit{softwares} maliciosos, são um grande problema para sistemas de informação, sua existência ou execução tem consequências negativas ou involuntárias. Os \textit{malwares} mais conhecidos são os vírus, worms e trojans.

 É importante entender o funcionamento e o comportamento desses códigos maliciosos para, a partir daí, buscar soluções contra esse ataque. Existe dois tipos de análise: análise estática, requer uma verifica linha a linha do código malicioso, geralmente o código não está disponível e até mesmo se estiver, o autor do \textit{malware} muitas vezes ofusca o código, tornando esse tipo de análise difícil. Por outro lado, existe a análise dinâmica, o analista monitora a execução e o comportamento do \textit{malware}, esse tipo de análise é imune a ofuscação de código \cite{encycrypt}.

 O Vírus é um programa que se propaga inserindo cópias de si mesmo e se tornando parte de outros programas e arquivos. Para dar continuidade ao processo de infecção, o vírus depende da execução do programa ou arquivo hospedeiro. O principal meio de propagação desse tipo de \textit{software} malicioso são as mídias removíveis, como, por exemplo, pen-drives \cite{certs-malwares}.

O Worm é um \textit{malware} que se propaga através de e-mails, sites ou \textit{software} baseados em rede, explorando as vulnerabilidades das aplicações. Uma das principais características desse tipo de \textit{software} é a propagação automática, ou seja, sem a intervenção do usuário \cite{detectingworm}. 

O Trojan ou Cavalo de Troia são programas que precisam ser explicitamente executados para serem instalados no computador. Esse \textit{malware} se disfarça de um programa benigno, por exemplo, cartões virtuais animados, álbuns de fotos, jogos e protetores de tela que ao serem executados o trojan é instalado sem o consentimento do usuário. No entanto, o atacante, após invadir um computador, pode instalar o trojan  alterando as funções já existentes de programas para executarem ações maliciosas \cite{certs-malwares}.

 \section{Ferramentas para Avaliação de Segurança} \label{sec:ferramentas}
 %Nmap, Metasploit, Pytbull

 Nessa seção será descrito as ferramentas auxiliares utilizadas para geração de ataques \autoref{sec:ataques} com objetivo de testar e validar as configurações das ferramentas de IDPS estudas. 

 \subsection{Nmap} \label{sec:nmap}

 O Nmap é uma ferramente de código aberto utilizada para auditoria de segurança e descoberta de rede. A ferramenta é capaz de determinar quais \textit{hosts} estão disponíveis na rede, quais serviços cada \textit{host} está oferecendo, incluindo nome e versão da aplicação, o sistema operacional usado, dentre outras características.  

 Muitos administradores de sistemas utilizam o Nmap para tarefas rotineiras como, criação de inventário de rede, gerenciamento de serviços, visto que é de suma importância manter os mesmos atualizados e monitoramento de \textit{host}.

 Diversos parâmetros podem ser utilizados com o Nmap, possibilitando realizar varreduras das mais variadas maneiras, dependendo do tipo desejado. A lista completa de opções podem ser consultadas na documentação oficial que vem junto da ferramenta ou no site do projeto \cite{nmap}. 

 Na execução do Nmap, o que não for opção ou argumento da opção é considerado especificação do \textit{host} alvo. O alvo pode ser um ou vários, usando uma notação de intervalo por hífen ou uma lista separada por vírgula. Os \textit{hosts} alvos também podem ser definidos em arquivos.

 O resultado do Nmap é uma tabela de portas e seus estados (\autoref{fig:nmap-exemplo}). As portas podem assumir quatro estados, temos: aberto (\textif{open}), significa que existe alguma aplicação escutando conexões; filtrado (\textit{filtered}), há um obstáculo na rede, podendo ser algum \textit{firewall}, que impossibilita que o Nmap determine se a porta está aberta ou fechada; fechado (\textit{closed}), não possui aplicação escutando na porta; não-filtrado (\textit{unfilterd}), a porta responde requisição porém o Nmap não consegue determinar se estão fechadas ou abertas \cite{nmap}

 \begin{figure}[htb]
  \centering
  \includegraphics[scale=.5]{nmap.png}
  \caption{Exemplo de saída do Nmap}
  \label{fig:nmap-exemplo}
 \end{figure}

 \subsection{Metasploit Framework} \label{sec:metasploit}

 O Metasploit é um \textit{framework} de código aberto cujo principio básico é desenvolver e executar \textit{exploit} contra alvos remotos e fornecer uma lista de vulnerabilidades existentes no alvo. É uma ferramenta que combina diversos \textit{exploits} e payloads dentro de um local, ideal para levantamento de segurança de serviços e testes de penetração \cite{metasploit:yash}.  

 O Metasploit possui uma biblioteca divida em três partes: \textbf{Rex}: É a biblioteca fundamental, a maioria das tarefas executadas pelo \textit{framework} usaram essa biblioteca; \textbf{MSF Core}: É o \textit{framework} em si, possui, por exemplo, gerenciador de módulos e a base de dados; \textbf{MSF Base}: Guarda os módulos, sejam eles, \textit{exploit}, \textit{encoders} (ferramentas usadas para desenvolver o \textit{payloads}) e os \textit{payloads}. Além disso, são guardadas informações de configuração e sessões criadas pelos \textit{exploits}. A arquitetura é mostrada com mais detalhes na \autoref{fig:metasploit-arquitetura}. 

 Os módulos são divididos da seguinte maneira: Payload: são código executados no alvo remotamente; Exploit: explora \textit{bugs} ou vulnerabilidade existente em aplicações do alvo; Módulos Auxiliares: usado para escanear as vulnerabilidades e executar várias tarefas; Encoder: codifica o \textit{payload} para evitar qualquer tipo de detecção pelo anti vírus.

 \textbf{Interface}: Disponibiliza a parte gráfica para o usuário;

 \begin{figure}[!htb]
  \centering
  \caption{Arquitetura do Metasploit}
  \includegraphics[scale=.6]{metasploit_arquitetura.png}
  \legend{}
  \label{fig:metasploit-arquitetura}
 \end{figure}

 \subsection{Pytbull} \label{sec:pytbull}

 O Pytbull é um \textit{framework} para teste de IDPS, capaz de determinar a capacidade de detecção e bloqueio do mesmo, além de fazer uma comparação entre diversas soluções e verifica as configurações \cite{pytbull}. O \textit{framework} Pytbull possui cerca de 300 testes agrupados em 11 módulos, temos:

 \begin{alineas}
  \item \textbf{badTraffic}: pacotes não compatíveis com a RFC são enviados para o servidor para testar como os pacotes são processados; 
  \item \textbf{bruteForce}: testa a capacidade do IDPS de rastrear ataques de força bruta;
  \item \textbf{clientSideAttacks}: usa um \textit{shell} reverso para fornecer ao servidor instruções para baixar arquivos maliciosos; 
  \item \textbf{denialOfService}: testa a capacidade do IDPS de proteger contra tentativas de DoS; 
  \item \textbf{evasionTechniques}: testa a capacidade do IDPS de detectar técnicas de evasão; 
  \item \textbf{fragmentedPackets}: várias cargas úteis fragmentadas são enviadas ao servidor para testar sua capacidade de recomposição e detectar os ataques; 
  \item \textbf{ipReputation}: testa a capacidade do servidor detectar tráfego de servidores com reputação baixa;
  \item \textbf{normalUsage}: cargas úteis que correspondem a uso normal; 
  \item \textbf{pcapReplay}: permite reproduzir arquivos pcap; 
  \item \textbf{shellCodes}: envia \textit{shellcodes} para o servidor na porta 21/ftp testando a capacidade de detectar e/ou bloquear o mesmo; 
  \item \textbf{testRules}, testa a base de assinaturas configuradas no servidor IDPS.
 \end{alineas}

 Existem basicamente 5 tipos de testes: socket, abre um \textit{socket} em uma porta e envia o \textit{payload} para o alvo remoto na porta especificada; command, envia um comando para alvo remoto com a função python subprocess.call(); scapy, envia cargas úteis especificas baseadas na sintaxe de Scapy; client side attacks, usa um \textit{shell} reverso no alvo remoto e envia comandos para serem processados no servidor; pcap replay, permite reproduzir tráfego com base em arquivos de pcap.

 \begin{figure}[htb]
  \centering
  \caption{Arquitetura do \textit{framework} Pytbull}
  \includegraphics[scale=.4]{arquitetura_pytbull.png}
  \legend{}
  \label{fig:pytbull}
 \end{figure}

 \section{Conclusão}
 %Este capítulo apresentou

\chapter{Sistemas de Detecção e Prevenção de Intrusão} \label{ch:idps}

Os sistemas de detecção e prevenção de intrusão (\textit{Intrusion Detection and Prevention System} - IDS/IPS) são ferramentas de importância reconhecida pela comunidade da segurança da informação. Nesse capítulo, vamos apresentar os principais conceitos relacionados a IDS e IPS, uma breve descrição do funcionamento e classificação, para melhor entendimento das ferramentas que iremos apresentar e avaliar em um ambiente de real.

\section{Definições de IDS/IPS} \label{sec:ipds-definicoes}

\textit{Intrusion Detection Systems} (IDS) ou Sistemas de Detecção de Intrusão (SDI) são ferramentas utilizadas para monitoramento de eventos que ocorrem em redes e sistemas computacionais, analisando sinais de possíveis ataques que podem levar a uma violação das politicas de segurança da organização, alertando os administradores do sistema que estes eventos estão ocorrendo. 

O \textit{Intrusion Detection Systems} (IPS) ou Sistema de Prevenção de Intrusão (SPI) possui todas as funcionalidades do IDS com uma diferença, ele é capaz de deter os incidentes, minimizando os impactos causados por sistemas comprometidos \cite{mukhopadhyay01}.

%localizar a referencia
Os IDS's são compostos basicamente por quatro componentes, temos: 
\begin{alineas}
\item \textbf{Sensor ou Agente}: responsável pelo monitoramento e analise do trafego capturado; 
\item \textbf{Base de Dados}: usado como repositório das informações de eventos detectados pelo sensor e que posteriormente serão processados;
\item \textbf{Gestor}: é o dispositivo central que recebe, analisa e gerencia as informações de eventos vindo do sensor; 
\item \textbf{Console}: é uma interface para administração e monitoramento das atividades.
\end{alineas}

\section{Tipos de Sistemas de Detecção e Prevenção de Intrusão} \label{sec:idps-tipos}

Os IDPS's são classificados de acordo com o local onde o sensor é instalado, \textit{Host Based Intrusion Detection Systems} (HIDS) e \textit{Network Based Intrusion Detection Systems} (NIDS), e a técnica utilizada para o monitoramento, baseado em assinaturas e anomalias \cite{nagahama2012ipsflow}.

\subsection{Sistemas de Detecção de Intrusão Baseados em Host (HIDS)}

Em um HIDS o sensor é instalado no \textit{host}, monitorando as informações contidas na própria máquina. Esse tipo de IDS não observa o tráfego que passa pela rede (somente o trafego que passa pela placa de rede do \textit{host}), seu uso volta-se a verificação de informações relativas aos eventos e registros de logs e sistemas de arquivos (permissão, alteração, acesso a arquivos não autorizados) \cite{nagahama2012ipsflow}.  

As vantagens do HIDS são: 

\begin{alineas}
\item Evita a execução de códigos maliciosos;
\item Bloqueia tráfego de entrada e saída contendo ataques e uso não autorizado de protocolos e programas;
\item Evita que arquivos possam ser acessados, modificados e deletados impedindo a instalação de \textit{malwares} e ataques envolvendo acesso inapropriado a arquivos;
\end{alineas}

Por outro lado, o HID possui alguns desvantagens como \cite{scarfone01}:  

\begin{alineas}
\item Difícil instalação e manutenção;
\item Interfere no desempenho do \textit{hosts};
\item Demora para identificar eventos consequentemente a resposta ao incidente terá um atraso.
\end{alineas}

\subsection{Sistemas de Detecção de Intrusão Baseados em Rede (NIDS)}

No NIDS, o sensor é instalado na rede e a interface de rede atua em um modo especial chamado ``promíscuo'', tendo a capacidade de capturar o tráfego mesmo que os pacotes não sejam destinados ao sensor. Dessa forma, o NIDS monitora e analisa todo o trafego no segmento da rede, detectando atividades maliciosas, como ataques baseados em serviço, \textit{portscans}, entre outros, além de detectar se algum usuário legítimo está fazendo mau uso da rede \cite{nagahama2012ipsflow}.

Quanto a localização o NIDS pode ser classificado como passivo ou ativo. No modo passivo (\autoref{fig_nids-ativo}), o IDS monitora copias dos pacotes da rede que passam pelo \textit{switch} ou \textit{hub} onde está conectado, ficando limitado somente a gerar notificações quando encontrado algum tráfego malicioso. 

No entanto, no modo ativo (\autoref{fig_nids-passivo}), o IDS é instalado da forma que o tráfego da rede passe através do sensor parecendo com o fluxo de dados associado com um \textit{firewall}. Dessa forma, ele é capaz de parar ataques bloqueando o fluxo malicioso. 

É necessário uma analise minuciosa na instalação de um IDS ativo pois um mal dimensionamento de \textit{hardware} pode degradar a rede, adicionando atrasos excessivos aos pacotes.

As principais vantagens do um NIDS são: 

\begin{alineas}
\item São independentes de plataformas;
\item Não interfere no desempenho do \textit{host};
\item Fácil implantação e transparente para o atacante.
\end{alineas}

Dentre as desvantagens, temos:

\begin{alineas}
\item Pode adicionar retardados aos pacotes quando instalado no modo ativo;
\item Dificuldade de tratar dados de redes de alta velocidade;
\item Trata apenas segmentos de rede;
\item Dificuldade de tratar dados criptografados.
\end{alineas}

%Devido a grande heterogeneidade de dispositivos e sistemas operacionais na rede, a utilização desse tipo de IDS torna a administração mais simples se comparados com o HIDS.

\begin{figure}[htb]
 \label{fig_nids-arquitetura}
 \centering
 \begin{minipage}{0.4\textwidth}
  \centering
  \caption{Exemplo de arquitetura de NIDS passivo} \label{fig_nids-passivo}
  \includegraphics[scale=.55]{nids_passivo.png}
  \legend{Fonte: Autoria própria}
 \end{minipage}
 \hfill
 \begin{minipage}{0.4\textwidth}
  \centering
  \caption{Exemplo de Arquitetura de NIDS ativo} \label{fig_nids-ativo}
  \includegraphics[scale=.55]{nids_ativo.png}
  \legend{Fonte: Autoria própria}
 \end{minipage}
\end{figure}

\subsection{Sistema de Detecção de Intrusão Distribuídos}

A função de um Sistema de Detecção de Intrusão Distribuído (SDID) é de gerencia. Os sensores (pode ser NIDS, HIDS ou a combinação de ambos), localizados remotamente, reportam os alertas para um centralizador. Os \textit{logs} de ataques são, periodicamente, enviados para a estação de gerenciamento, armazenando em uma base única e centralizada, além disso, novas assinaturas de ataques podem ser enviadas para os sensores \cite{snort:andrew}.   

\begin{figure}[!htb]
  \centering
  \caption{Sistema de Detecção de Intrusão Distribuído} \label{fig_dids}
  \includegraphics[scale=0.7]{dids.png}
  \legend{Fonte: Autoria própria}
\end{figure}

Na \autoref{fig_dids} mostra um SDID composto por dois sensores e um estação de gerenciamento centralizado. O sensor NIDS 1 e NIDS 2 estão operando em modo \textit{promiscuos} e está protegendo segmentos de rede. É recomendando que a conexão entre os sensores e o centralizado seja feita por uma rede privada, em caso de utilização de rede públicas, recomenda-se adicionar uma camada de segurança, como criptografia, ou VPN.

\subsection{Formas de Detecção} \label{sec:idps-formas}

Quanto a técnica de monitoramento utilizado, o IDS pode ser baseados em assinaturas ou anomalias. IDSs baseados em assinaturas compara os pacotes com uma base de assinaturas de ataques previamente conhecidos e reportados por especialistas, cada assinatura identifica um ataque \cite{nagahama2012ipsflow}.  

As vantagens de um IDS baseados em assinaturas são:

\begin{alineas}
\item Usa pouco recurso de \textif{hardware} do servidor;
\item Possui, de certa forma, um rápido processamento.
\end{alineas}

Dentre as desvantagens temos:

\begin{alineas}
\item Exige uma atualização constante da base de assinaturas;
\item Para a geração de uma base própria, a equipe precisa de um alto conhecimento técnico;
\item Possui altos índices de falsos positivos e negativos.
\end{alineas}

Os IDS baseados em anomalias, procuram determinar um comportamento normal na fase de aprendizagem do sistema computacional ou rede e sempre que existir um desvio desse padrão alertas são gerados. 

Possui a vantagem de detectar novos ataques sem necessariamente conhecer a fundo a intrusão através dos desvios de comportamento. Porém, tem como desvantagem a geração de um grande número de falsos alertas em decorrência a modificações na rede ou \textit{host} nem sempre representar um tráfego malicioso. 

\section{Principais Ferramentas de IDS} \label{sec:idps-ferramentas}

Nesse capitulo, será apresentado as ferramentas de IDPS analisadas. A escolha dessas ferramentas deu-se devido ser de código aberto e de livre uso, e também, pela sua popularidade diante da comunidade de segurança da informação.

\subsection{Snort} \label{sec:snort}

O Snort é um sistema de detecção e prevenção de intrusão de código fonte aberto escrita na linguagem de programação C bem conhecido pela comunidade da segurança da informação. Seu primeiro \textit{release} foi lançado em 1998 e desde então passa por constantes revisões e aperfeiçoamentos, com o passar dos anos se tornou o IDS mais utilizado no mundo. Ele combina análise baseada em assinaturas e anomalias, podendo operar em três modos: \textit{sniffer}, \textit{packet logger} e de sistema de detecção de intrusão (NIDS) \cite{snort:manual}.

No modo \textit{Sniffer}, o Snort captura os pacotes e exibi as informações no console de forma continua. No modo \textit{Packet Logger}, além de capturar o tráfego, o Snort escreve essas informações em arquivos (chamados de logs) que são armazenados no disco. Por fim, o \textit{Network Intrusion Detection System} - NIDS, sendo o modo mais complexo e completo, permitindo capturar e analisar os pacotes de rede em tempo real \cite{snort:manual}.

Existe quatro componentes no Snort: O \textit{sniffer}, o pré-processador, o motor de detecção e módulos de saída. A \autoref{fig_snort-componentes} mostra a arquitetura e disposição dos componentes \cite{snort:andrew}.

\begin{figure}[!htb]
  \centering
  \caption{Arquitetura do Snort} \label{fig_snort-componentes}
  \includegraphics[scale=0.6]{snort_componentes}
  \legend{Fonte: Autoria própria}
\end{figure}

O pré-processador, o motor de detecção e os componentes de alerta do Snort são todos \textit{plugins}. Os \textit{Plugins} são programas escritos em conformidade com a API de \textit{plugins} do Snort. Esses programas são usados no core do Snort, mas eles são separados para que as modificações feitas no \textit{core} sejam mais confiáveis e mais fáceis de realizar \cite{snort:andrew}.

O \textit{sniffer} é um dispositivo (\textit{software} ou \textit{hardware}) usado para ver o trafego passante em algum segmento de rede. No caso da Internet, consiste geralmente de trafico IP (composto por diferentes protocolos de alto nível como, TCP, UDP, ICMP, protocolos de roteamento e IPSec). Os pacotes são analisado, interpretados e exibidos de uma forma legível para os humanos.

Um \textit{sniffer} tem os seguintes usos:

\begin{alineas}
\item Analisador de rede e resolução de problemas;
\item Analisador de performance e avaliação comparativa;
\item Capturar senhas em texto plano e outros dados sensíveis.
\end{alineas}

Assim como qualquer outra ferramenta de rede, os \textit{sniffers} podem ser usados tanto para o bem quanto para o mal. Então, criptografar o trafego de rede previne que pessoas sejam capazes de lerem os pacotes capturados \cite{snort:andrew}.

O pré-processador pega o pacote bruto e faz uma checagem utilizando um determinado \textit{plugin}. Esses \textit{plugins} verificam se o pacote tem um tipo particular de comportamento, uma vez determinado, o pacote é enviado para o motor de detecção caso contrário é descartado.

Na \autoref{fig_snort_preprocessor}, pode-se ver como o pré-processador utiliza \textit{plugins} para chegar pacotes. O Snort suporta muitos tipos de pré-processadores, cobrindo vários protocolos comumente usados como, IP \textit{fragmentation handling}, \textit{port scanning} e controle de fluxo.

\begin{figure}[!htb]
  \centering
  \caption{Uso de \textit{plugins} no pré-processador} \label{fig_snort_preprocessor}
  \includegraphics[scale=0.8]{snort_preprocessor.png}
  \legend{Fonte: Autoria própria}
\end{figure}

O uso de \textit{plugins} é uma característica muito útil para o IDS, pois os \textit{plugins} podem ser ativados e desativadas a medida do necessário, otimizando a utilização dos recursos computacionais e geração de alertas \cite{snort:andrew}.

Os pacotes, após passarem por todos os pré-processador, são entregues para o motor de detecção. O motor de detecção pega esses dados e faz uma checagem utilizando uma base de regras pré-configurado pelo administrador. Se a regra for compatível com os dados do pacote, eles são enviado para o processador de alertas, caso contrário, são descartados \cite{snort:andrew}.

Na \autoref{fig_snort_detecção}, temos os pacotes saindo dos pré-processadores e chegando no motor de detecção. No motor de detecção há uma base de regras configurada, os dados dos pacotes são comparados com as assinaturas da base, se coincidirem, uma ação é tomada, caso contrário, o pacote é descartado.

\begin{figure}[!htb]
  \centering
  \caption{Motor de Detecção do Snort} \label{fig_snort_detecção}
  \includegraphics[scale=0.8]{snort_detection.png}
  \legend{Fonte: Autoria própria}
\end{figure}

A base de regras é um conjunto de assinaturas de ataques conhecidos e catalogados. As regras são escritas em formato texto em uma única linha e constituídas por duas partes: 

\begin{alineas}
\item \textbf{Cabeçalho}: São definidos que ações serão tomadas, tipo de pacote (TCP, UDP, ICMP, etc), o IP de origem e destino e porta;
\item \textbf{Opções}: É o conteúdo do pacote que faz ele ser compatível com a regra.
\end{alineas}

Dentre as ações que podem ser tomadas temos:

\begin{alineas}
\item \textbf{\textit{Activation}}: Alerta e chama regra do tipo \textit{dynamic};
\item \textbf{\textit{Dynamic}}: permanece inativa até ser ativado por uma regra \textit{activate}, registrando o tráfego;
\item \textbf{\textit{Alert}}: Gera um alerta usando um método selecionado e então registra os pacotes e dados;
\item \textbf{\textit{Pass}}: Ignora os pacotes;
\item \textbf{\textit{Log}}: Registra e não alerta.
\end{alineas}

Abaixo temos um exemplo de regra.

\begin{lstlisting}
alert icmp any any -> any any (msg:"Ping suspeito"; 
sid:1; resp:icmp_all;)
\end{lstlisting}

Com a regra acima, o Snort gerará um alerta de qualquer pacotes ICMP que estiver passando de qualquer máquina e porta origem (\textbf{any any}) para qualquer máquina e porta destino (\textbf{any any}) e enviará pacotes ICMP para a máquina de origem com as mensagens \textit{host unreachable; network unreachable}.

\subsection{Suricata} \label{sec:suricata}
\section{Conclusão} \label{sec:idps-conclusao}
%Ex: Este capítulo apresentou...


\chapter{Detecção de Intrusão em um Cenário Real}
 %- Em cada capitulo adicionar um texto introdutório

 Este capítulo esta organizado da seguinte forma: A próxima seção apresenta o cenário de testes, descrevendo características gerais da rede selecionada para os teste. Na seção \ref{sec:infraestrutura} será abordado a infraestrutura usada para os testes, ferramentas utilizadas e as configurações feitas. Na seção \ref{sec:testes} será descrito os testes realizados com suas respectivas justificativas. Na seção \ref{sec:resultados} será apresentado os resultados esperados e obtidos, problemas encontrados e a comparação das ferramentas e por último, na seção \ref{sec:conclusão}, uma breve conclusão.

 \section{Metodologia dos Testes}
 % (Adicionar o escopo dos testes): descrever o ambientes,
 \subsection{Cenário de Testes} \label{sec:cenário}
 %Descrever o cenário. Ex: Uma rede com XXX usuários; Os usuários utilizam
 %diferentes ferramentas,...
 %Adicionar gráficos de utilização da rede

 A rede selecionada para ser monitorada tem os valores especificados na tabela. Podemos verificar que em um determinado período do dia o pico de trafego chega a 107,25 Mbps, valores considerados ideais para o experimento, inclusive para tentar validar os recursos alocados. Figura \ref{fig:ilc}

 Em um primeiro momento, selecionou-se uma rede

 \subsection{Infraestrutura Definida para Testes} \label{sec:infraestrutura}
 %Descrever o que foi utilizado:
 %Ex: Equipamentos - Servidores, ....
 %    Configuração
 %    Ferramentas - Snort (referenciar capítulo), ...

 No ambiente de teste foi usado uma máquina Dell com 134 Megabytes (MB) de memória RAM e 40 núcleos. Usou-se XenServer \cite{xenserver} versão 7, sistema operacional (SO) \textit{opensource} da Citrix voltado para virtualização. Foram testados outros SOs porém somente o XenServer possuía, na época da instalação do ambiente, \textit{firmware} da placa de rede do \textit{host} compatível e que funcionava com instabilidade. Outro fator que pesou na escolha do SO foi a experiência que tinha com a plataforma e por existir uma interface para gerencia chamada XenCenter que roda no Windows. Uma alternativa \textit{opensource} desse software é o OpenXenManager \cite{openxenmanager}.

 No primeiro momento, foi instalado uma máquina virtual com o sistema operacional Debian 7.11 \textit{codename} Wheezy \cite{debianwheezy}, uma distribuição linux com uma proposta de ser totalmente livre, usada como base para instalação de outras máquinas utilizando o recurso de \textit{snapshot}, uma cópia de uma máquina virtual rodando em um certo momento, do XenServer. O uso desse recurso foi necessário para criar um ambiente igual para os IDSs.

 Foi alocado 8 MB memória RAM, 4 processadores e 100 Gigabytes(GB) de espaço em disco para o \textit{snapshot}. Esses valores foram definidos com base em um estudo \cite{mikelococo} que considerava vários fatores, como largura da rede, localização do IDS e versão, tipo do capturador de tráfego e tamanho da base de assinaturas para dimensionar os recursos de memória e processamento, aplicado especificamente ao Snort. A mesma regra foi aplicada ao Suricata.

 Para o \textit{host} conseguir pegar o pacotes destinados a rede escolhida para ser monitorada foi necessário uma configuração de espelhamento no roteador B (Figura \ref{fig:infra-ambiente}) que consiste na copia dos pacotes que saem pela porta dessa rede no roteador para a porta conectada no \textit{host} que possui uma largura de banda de 10 Gigabits. A interface de rede do \textit{host} precisou ser configurada no modo \textit{promisc}.

 Posteriormente criou-se três máquinas virtuais, duas usadas para instalação dos IDSs (Suricata e Snort) e a terceira para instalação das ferramentas usadas para simular ataques a rede. Optou-se pela instalação do sistema Kali Linux \cite{kalilinux} para geração de ataques pois nele existe várias ferramentas nativas para testes de penetração e auditoria de segurança. A infraestrutura final pode ser visualizada na Figura \ref{fig:infra-ambiente}.

 \begin{figure}[!htb]
  \centering
  \includegraphics[scale=.5]{infra.png}
  \caption{Infraestrutura do ambiente de teste}
  \label{fig:infra-ambiente}
 \end{figure}

 Para coleta das informações de uso de recurso de hardware como memória, processamento e I/O das máquinas com os IDSs foi usado o \textit{daemon} Collectd \cite{collectd}. Outra opção para esse fim é a utilização de um servidor de monitoramento com o Zabbix \cite{zabbix}. A ideia de usar duas ferramentas para analise é fazer um comparativo e validar as informações coletadas.

O formato usado para facilitar a análise do \textit{logs} foi JavaScript Object Notation (JSON), um formato simples, leve e de fácil leitura. O Motor de Saída do Suricata já tem suporte a esse tipo de formato o que não acontece no Snort. Para tal, usou-se o IDSTools \cite{py-idstools}, uma coleção de bibliotecas na linguagem python que trabalha para auxiliar o IDS, compatível com as ferramentas estudas. Dentre os utilitários presentes nessa coleção, temos o idstools-u2json, que converte, de forma continua, arquivo no formato unified2, uma das saídas disponível no Snort, para o formato JSON.

Para analisar os \textit{logs}, usou-se uma infraestrutura que combina três ferramentas, o Kibana \cite{kibana}, uma plataforma de análise e visualização desenhada para trabalhar com os índices do Elasticsearch \cite{elasticsearch}, a grosso modo, podemos dizer que ela é uma interface gráfica para o Elasticsearch. O Elasticsearch, um motor de busca e análise altamente escalável, capaz de armazenar, buscar e analisar uma grande quantidade de dados em tempo próximo ao real. Por ultimo, o Logstach \cite{logstach}, um motor de coleta de dados em tempo real, unificando os dados de diferentes fontes dinamicamente, normalizando-os nos destinos escolhidos (Figura \ref{fig:logstach}). Dessa forma centralizou-se os \textit{logs}, facilitando a visualização das ocorrências dos IDSs. 

\begin{figure}[!htb]
 \centering
 \includegraphics[scale=.4]{logstach.png}
 \caption{Busca e união dos dados de diferentes fontes}
 \label{fig:logstach}
\end{figure}
section{Testes Realizados} \label{sec:testes}
%Quais os testes realizados com justificativa ?
%Descrição dos testes. Quais os testes foram realizados ?
Os testes realizados são simulações de passos que uma pessoa má intencionada iria tomar para alguma tentativa de invasão, entende-se por invasão, qualquer tipo de violação e alteração não autorizada de um serviço ou \textit{host}.

O passo inicial seria um estudo do alvo por engenharia social, analisando as pessoas que trabalharam no organização, enviando spam e phishing na tentativa de capturar dados como senhas de acesso. Posteriormente, verificando os serviços que o alvo oferece e observando (\textit{sniffando}) a rede, a procura de alguma senha desprotegida (não criptografada). Esse passo inicial não será aplicados nos testes pois seria impossível o IDS detectar.

O passo seguinte seria um estudo e mapeamento da rede, a procura de um \textit{host} vulnerável. A ferramenta escolhida para essa finalidade é o Nmap \ref{sec:nmap}. No primeiro teste de Scan, usou-se o parâmetro "-F", habilitando a modo \textit{fast} do Nmap. Nesse modo, são verificadas apenas a portas especificadas no arquivo nmap-services, na instalação padrão esse arquivo vem com 27372 portas descritas. Isso é muito mais rápido que verificar todas as 65535 portas possíveis em um \textit{host}.

nmap -F 200.239.82.0/24

O segundo teste, usou-se o parâmetro '-sV' do Nmap. Essa opção habilita a descoberta de versões, tentando determinar os protocolos de serviços, o nome da aplicação, o número da versão, o nome do \textit{host}, tipo de dispositivo, sistema operacional usado, entre outras informações. Essas informações são de grande valor pois, a partir delas, pode-se explorar vulnerabilidades conhecidas de uma determinada versão de um serviço \cite{nmap}.

nmap -sV 200.239.82.0/24

De posse de um alvo em potencial, próximo passo seria rodar um scan de vulnerabilidade, em busca de brejas já conhecidas, e que, geralmente por descuido do administrador, não foi fechada. Para esses testes usou-se o \textit{framework} Metasploit \ref{sec:metasploit} nativo do sistema operacional Kali Linux.

\section{Resultados} \label{sec:resultados}
%Resultados esperados e obtidos
%Quais os resultados dos testes ??
%Comparação das ferramentas
%Problemas encontrados
\section{Conclusão} \label{sec:conclusão}
\section{Métricas de Comparação}
%Consumo dos Recursos de Hardware (Memória, Processamento)
%Taxa de Detecção
%Número de Falsos Positivos/Negativos

\chapter{Considerações Finais e Trabalhos Futuros} \label{ch:considerações}


\phantompart
\postextual
\bibliography{bib/dissertacao}
\phantompart
\printindex
\end{document}
