\chapter{Introdução} \label{ch:introdução}

A Internet é um conjunto de redes físicas heterogênea (uma variedade de dispositivos conectados, \textit{smartphones}, \textit{desktops}, \textit{notebooks}, servidores, \textit{switches}, roteadores, entre outros) funcionando como uma rede lógica única de alcance mundial. O grande e contínuo crescimento da Internet gerou um aumento da sua complexidade, que a expõe a diversas vulnerabilidades. 

A todo momento, novos ataques ou mesmo variações de ataques já existentes surgem e são lançados a várias redes indiscriminadamente em busca de vulnerabilidades. Em 2015, foram reportados ao Centro de Estudos, Respostas e Tratamento de Incidentes de Segurança no Brasil (CERT.br) cerca de 647.112 incidentes, em 2017, esse número aumentou, chegando a 833.775 (\autoref{fig:cert}) \cite{estatistica:cert.br}.

\begin{figure}[!htb]
 \centering
 \caption{Estatísticas de ataques reportadas ao CERT.br}
 \includegraphics[scale=.6]{incidentes-reportados.png}
 \legend{Fonte: \cite{tipos-ataques:certs.br}}
 \label{fig:cert}
\end{figure}

O CERT.br é o grupo de resposta a incidentes de segurança para a internet brasileira, mantido pelo Comitê Gestor da Internet no Brasil. Atua na notificação e tratamento de incidentes de segurança dando apoio no processo de resposta. Além disso, faz um trabalho de conscientização e treinamento sobre problemas de segurança no Brasil.

Paralelamente ao CERT.br temos o Centro de Atendimento a Incidentes de Segurança (CAIS), mantido pela Rede Nacional de Ensino e Pesquisa (RNP). O CAIS é responsável por zelar pela segurança da rede Ipê (infraestrutura de rede dedicada à comunidade brasileira de ensino superior), detectando, resolvendo e prevenindo incidentes de segurança. Além disso, tem o papel de orientar (através de publicações de cartilhas) e disseminar boas práticas de segurança da informação, educando e conscientizando usuários de todos os níveis sobre os principais riscos em segurança da informação \cite{cais}.

Desde 2008, todas a fraudes identificadas pelo CAIS estão sendo ordenadas e disponibilizadas para consulta (\autoref{fig:cais}). Adicionalmente, são enviados alertas através de uma lista quando uma fraude mostra-se, particularmente, perigosa aos usuários e computadores.

\begin{figure}[!htb]
 \centering
 \caption{Estatísticas de incidentes reportados ao CAIS}
 \includegraphics[scale=.7]{cais-incidentes-reportados.png}
 \legend{Fonte: \cite{cais}}
 \label{fig:cais}
\end{figure}

As empresas, de qualquer segmento e tamanho, devem ter o trabalho de manter os ativos seguros e isso vai além da utilização de anti-vírus nos computadores. Ter uma política de atualização de \textit{software} em estações de trabalho e servidores, aliados com boas práticas nas configurações de serviços, dificultam a exploração de vulnerabilidades. 

A utilização de \textit{firewall}, não pode ser encarado com uma solução definitiva, passando uma falsa sensação de segurança, pois muitas portas legítimas podem estar vulneráveis, como acontece, por exemplo, com a porta 80 que hospeda \textit{websites} vulneráveis \cite{analisenessus:cleriston}.

Diante desse cenário, torna-se cada vez mais importante para o administrador de rede e/ou segurança da informação o uso de ferramentas de IDPS, permitindo identificar e tratar de forma automatizada os incidentes de segurança.

Para implementa tais ferramentas em uma rede, deve-se levar em consideração a flexibilidade e a administração simplificada para não resultar que a empresa/instituição fique na dependência de um único fabricante ou fornecedor da solução. Além disso, a ferramenta deve ter um bom desempenho e eficiência para não passar a falsa sensação de proteção ou degradar o desempenho da rede.

\section{Motivação} \label{sec:motivação} 

A motivação desse trabalho surgiu da necessidade de implantação de um Sistema de Detecção e Prevenção de Intrusão por parte da recém formada equipe de segurança da Universidade Federal do Pará (UFPA). Tal necessidade visa melhorar a segurança de modo geral da rede acadêmica da instituição. 

Assim, diante de uma pesquisa e devido questões orçamentárias, procurou-se ferramentas de uso gratuito e com bom desempenho, capaz de atender, de forma satisfatória, as necessidades da instituição. 

Como desdobramento, encontrou-se duas ferramentas, o Snort, que se encontra no mercado desde 1998, foi a primeira no seu segmento a realizar análise de tráfego em tempo real, e o Suricata, lançado em 2010, com uma arquitetura similar ao Snort, no entanto, utiliza \textit{multithreading}, visando melhorar ainda mais o desempenho.

\section{Objetivo Geral} \label{sec:objectivo-geral}

Este trabalho tem como objetivo geral, avaliar e comparar as soluções de código aberto de Sistemas de Detecção e Prevenção de Intrusão: Snort e Suricata. 

\section{Objetivos Específicos} \label{sec:objetivos-especificos}

Como desdobramento do objetivo geral, os seguintes objetivos específicos foram definidos:

\begin{alineas}
\item Apresentar conceitos sobre segurança da informação e sua importância em um ambiente corporativo;
\item Descrever problemas relacionados a ataques envolvendo redes de computadores;
\item Descrever as ferramentas, compreendendo requisitos, características, modos de atuação e funcionalidades;
\item Descrever o ambiente experimental;
\item Realizar experimentos e coletar dados para validar o funcionamento e a eficiência das ferramentas;
\end{alineas}

\section{Contribuições do Trabalho}

Este trabalho tem como principal contribuição apresentar um comparativo entre as principais ferramentas de IDS disponíveis. Em decorrência disso, ele contribui para a escolha de ferramentas desse tipo por equipes de segurança e auxilia na identificação de ameaças que venham afetar o funcionamento de sistemas computacionais.

\section{Metodologia} \label{sec:metodologia}

O método de pesquisa utilizado é o qualitativo, apoiando-se em técnica de coleta e análise de dados a partir de ferramentas. O estudo foi desenvolvido a partir de:

\begin{alineas}
\item Pesquisa bibliográfica: Os conceitos e referencias teóricos para entendimento do trabalho analisados foram: "Redes de Computadores", documentação e trabalhos referentes as ferramentas em estudos e a cartilha do CERT.br que descreve os ataques lançados as redes;
\item e a pesquisa de campo feita através da implementação de uma infraestrutura capaz de coletar dados, utilizando ferramentas auxiliares, sobre os objetos estudados.
\end{alineas}

A infraestrutura montada utiliza várias tecnologias, dentre elas temos virtualização, monitoramento de servidores, espelhamento de tráfegos de rede, etc. Para obter uma conclusão, os dados coletados foram comparados e ao final verificou-se quais das ferramentas obtiveram melhores resultados.

\section{Trabalhos Relacionados} \label{sec:trabalhos-relacionados}

O uso de ferramentas IDS \textit{opensource} Snort e/ou Suricata, importantes na área de segurança, já foi abordado e apresentou alguns resultados satisfatórios em pesquisas anteriores, por exemplo, nas pesquisas de Nagahama \textit{et al.} (2013), Murini \textit{et al} (2014) e Lopez \textit{et al.} (2014). 

No trabalho do Nagahama \textit{et al.} (2013), é usado redes definidas por \textit{softwares}, que desacopla os planos de controle e de dados, permitindo adaptar o funcionamento da rede de acordo com a necessidade de cada um. O \textit{software} utilizado na pesquisa foi o OpenFLOW. Tal uso, tem como objetivo mitigar a falta de integração do IDS com os equipamentos de rede como \textit{switches} e roteadores, o que limita a atuação destas ferramentas. 

O trabalho do Nagahama foi de fundamental importância para o entendimento dos conceitos relacionados a Sistemas de Detecção e Prevenção de Intrusão e sobre a ferramenta Snort. No entanto, ele não faz um comparativo de desempenho entre as ferramentas proposto nesse trabalho.

Já Murini \textit{et al} (2014), foi feito uma comparação de desempenho das ferramentas de IDS (Snort e Suricata), porém usou-se dados sintéticos fornecidos pela \textit{Defense Advanced Research Projects Agency} (DARPA) e devido ser uma base com ataques antigos, não houve resultados satisfatórios. Ao final, listou-se as vantagens e desvantagens existentes de cada ferramenta. No trabalho proposto, no entanto, serão usados dados mais próximo de um ambiente de produção de um rede acadêmica. 

Por fim, Lopez \textit{et al.} (2014), utiliza-se do BroFlow que possui uma série de vantagem, como, detecção de intrusão através de algoritmos simples, modular e flexível, reação imediata a um ataque descantando pacotes dos atacantes os mais próximo da origem. Dentre os resultados obtidos, destacam-se que a ferramenta conseguiu garantir o encaminhamento de pacotes legítimos na rede na taxa máxima do enlace e reduziu, em até dez vezes, o atraso na rede provocado pelo ataque.

\section{Organização do Trabalho} \label{sec:organização-do-trabalho}

O restante desse trabalho está organizado da seguinte forma.

No \autoref{ch:segurança}, são definidos conceitos sobre redes de computadores e segurança da informação, também são descritos os ataques comuns e ferramentas utilizadas para validar e avaliar as soluções de IDPS.

Em seguida, no \autoref{ch:idps}, a definição IDPS, os tipos existentes, as funcionalidades e descrição das ferramentas avaliadas: Snort e Suricata.

O \autoref{ch:cenário-real} detalhará o cenário real e a infraestrutura utilizada para os realização dos testes das ferramentas, os testes realizados e o resultados obtidos.

Por fim, no \autoref{ch:considerações}, as considerações finais e trabalhos futuros.
