\chapter{Sistemas de Detecção e Prevenção de Intrusão} \label{ch:idps}

Os sistemas de detecção e prevenção de intrusão (\textit{Intrusion Detection and Prevention System} - IDS/IPS) são ferramentas de importância reconhecida pela comunidade da segurança da informação. Nesse capítulo, vamos apresentar os principais conceitos relacionados a IDS e IPS, uma breve descrição do funcionamento e classificação, para melhor entendimento das ferramentas que iremos apresentar e avaliar em um ambiente de real.

\section{Definições de IDS/IPS} \label{sec:ipds-definicoes}

\textit{Intrusion Detection Systems} (IDS) ou Sistemas de Detecção de Intrusão (SDI) são ferramentas utilizadas para monitoramento de eventos que ocorrem em redes e sistemas computacionais, analisando sinais de possíveis ataques que podem levar a uma violação das politicas de segurança da organização, alertando os administradores do sistema que estes eventos estão ocorrendo. O \textit{Intrusion Detection Systems} (IPS) ou Sistema de Prevenção de Intrusão (SPI) possui todas as funcionalidades do IDS com uma diferença, ele é capaz de deter os incidentes, minimizando os impactos causados por sistemas comprometidos \cite{mukhopadhyay01}.

%localizar a referencia
Os IDS's são compostos basicamente por quatro componentes, temos: 
\begin{alineas}
\item \textbf{Sensor ou Agente}: responsável pelo monitoramento e analise do trafego capturado; 
\item \textbf{Base de Dados}: usado como repositório das informações de eventos detectados pelo sensor e que posteriormente serão processados;
\item \textbf{Gestor}: é o dispositivo central que recebe, analisa e gerencia as informações de eventos vindo do sensor; 
\item \textbf{Console}: é uma interface para administração e monitoramento das atividades.
\end{alineas}

\section{Tipos de Sistemas de Detecção e Prevenção de Intrusão} \label{sec:idps-tipos}

Os IDPS's são classificados de acordo com o local onde o sensor é instalado, \textit{Host Based Intrusion Detection Systems} (HIDS) e \textit{Network Based Intrusion Detection Systems} (NIDS), e a técnica utilizada para o monitoramento, baseado em assinaturas e anomalias \cite{nagahama2012ipsflow}.

\subsection{Sistemas de Detecção de Intrusão Baseados em Host (HIDS)}

Em um HIDS o sensor é instalado no \textit{host}, monitorando as informações contidas na própria máquina. Esse tipo de IDS não observa o tráfego que passa pela rede (somente o trafego que passa pela placa de rede do \textit{host}), seu uso volta-se a verificação de informações relativas aos eventos e registros de logs e sistemas de arquivos (permissão, alteração, acesso a arquivos não autorizados) \cite{}.  

As vantagens do HIDS são: 

\begin{alineas}
\item Evita a execução de códigos maliciosos;
\item Bloqueia tráfego de entrada e saída contendo ataques e uso não autorizado de protocolos e programas;
\item Evita que arquivos possam ser acessados, modificados e deletados impedindo a instalação de \textit{malwares} e ataques envolvendo acesso inapropriado a arquivos;
\end{alineas}

Por outro lado, o HID possui alguns desvantagens como \cite{scarfone01}:  

\begin{alineas}
\item Difícil instalação e manutenção;
\item Interfere no desempenho do \textit{hosts};
\item Demora para identificar eventos consequentemente a resposta ao incidente terá um atraso.
\end{alineas}

%s HIDS possuem algumas vantagens como evitar que alguns códigos sejam executados; bloqueia o tráfego de entrada e saída contendo ataques e uso não autorizado de protocolos e programas; evita que arquivos possam ser acessados, modificados e deletados impedindo o instalação de \textit{malware} e outros ataques envolvendo acesso inapropriado a arquivos. Por outro lado, um HIDS possui algumas limitações como, por exemplo, difícil instalação e manutenção; interferir no desempenho do \textit{host}; demora para identificar alguns eventos consequentemente a resposta a esses incidentes sofrerá um \textit{delay} \cite{scarfone01}.

\subsection{Sistemas de Detecção de Intrusão Baseados em Rede (NIDS)}

No NIDS, o sensor é instalado na rede e a interface de rede atua em um modo especial chamado ``promíscuo'', tendo a capacidade de capturar o tráfego mesmo que os pacotes não sejam destinados ao sensor. Dessa forma, o NIDS monitora e analisa todo o trafego no segmento da rede, detectando atividades maliciosas, como ataques baseados em serviço, \textit{portscans}, entre outros, além de detectar se algum usuário legítimo está fazendo mau uso da rede \cite{}.

Quanto a localização o NIDS pode ser classificado como passivo ou ativo. No modo passivo, o IDS monitora copias dos pacotes da rede que passam pelo \textit{switch} ou \textit{hub} onde está conectado, ficando limitado somente a gerar notificações quando encontrado algum tráfego malicioso. Enquanto no modo ativo, o IDS é instalado da forma que o tráfego da rede passe através do sensor parecendo com o fluxo de dados associado com um \textit{firewall}. Dessa forma, ele é capaz de parar ataques bloqueando o fluxo malicioso. 

Os NIDS possuem algumas vantagens como serem independentes de plataforma; não interfere no desempenho do \textit{host}; fácil implantação e transparente para o atacante. Por outro lado, possuem desvantagens como: pode adicionar retardados nos pacotes quando instalado no modo ativo, isso ocorre principalmente se houver um subdimensionamento do \textit{hardware}; dificuldade de tratar dados de redes de alta velocidade; quando em modo passivo, trata apenas o segmento da rede que o IDS esta instalado e dificuldade de tratar dados criptografados. Esse tipo de IDS é mais utilizado devido a grande heterogeneidade de dispositivos e sistemas operacionais disponíveis na rede, tornando a administração mais simples se comparados com o HIDS.

 Quanto a técnica de monitoramento utilizado, o IDS pode ser baseados em assinaturas ou anomalias. IDSs baseados em assinaturas compara os pacotes com uma base de assinaturas de ataques previamente conhecidos e reportados por especialistas, cada assinatura identifica um ataque. Tem como vantagem, usar poucos recursos do servidor e rápido processamento. Porém, as desvantagens são: exigi uma atualização constante da base de assinaturas; alto conhecimento para geração da base e possuir um alto índice de falsos positivos e negativos.

 Já os baseados em anomalias, procuram determinar um comportamento normal na fase de aprendizagem do sistema computacional ou rede e sempre que existir um desvio desse padrão alertas são gerados. Possui a vantagem de detectar novos ataques sem necessariamente conhecer a fundo a intrusão através dos desvio de comportamento. Porém existe a desvantagem de gerar um grande número de falsos alertas em decorrência a modificações na rede ou \textit{host} nem sempre representar um tráfego malicioso. 

\begin{figure}[htb]
 \label{fig:nids-arquitetura}
 \centering
 \begin{minipage}{0.4\textwidth}
  \centering
  \label{fig:nids-passivo}
  \caption{Exemplo de arquitetura de NIDS passivo}
  \includegraphics[scale=.55]{nids_passivo.png}
  \legend{Fonte: Autoria própria}
 \end{minipage}
 \hfill
 \begin{minipage}{0.4\textwidth}
  \centering
  \label{fig:nids-ativo}
  \caption{Exemplo de Arquitetura de NIDS ativo}
  \includegraphics[scale=.55]{nids_ativo.png}
  \legend{Fonte: Autoria própria}
 \end{minipage}
\end{figure}

\section{Principais Ferramentas de IDS} \label{sec:idps-ferramentas}
\subsection{Snort} \label{sec:snort}

O Snort é um sistema de detecção e prevenção de intrusão de código fonte aberto escrita na linguagem de programação C bem conhecido pela comunidade da segurança da informação. Seu primeiro \textit{release} foi lançado em 1998 e desde então passa por constantes revisões e aperfeiçoamentos, com o passar dos anos se tornou o IDS mais utilizado no mundo. Ele combina análise baseada em assinaturas e anomalias, podendo operar em três modos: \textit{sniffer}, \textit{packet logger} e de sistema de detecção de intrusão (NIDS) \cite{snortorgbr}.

No modo \textit{Sniffer}, o Snort captura os pacotes e exibi as informações no console. No modo \textit{Packet Logger}, além de capturar o tráfego, ele registrar essas informações em disco (arquivos de logs). E no modo NIDS, é o modo mais complexo, permite analise do pacotes de rede em tempo real.

Existe quatro componentes no Snort: O \textit{Sniffer}, o Pré-processador, o Motor de Detecção e Módulo de Saída. Os componentes são organizados de acordo com a figura \ref{snort-componentes} \cite{kohlenberg2007snort}.

 \begin{figure}[!htb]
   \centering
   \includegraphics[scale=0.6]{snort_componentes}
   \caption{Componentes do Snort}
   \label{snort-componentes}
 \end{figure}

 \subsection{Suricata} \label{sec:suricata}
 \section{Conclusão} \label{sec:idps-conclusao}
 %Ex: Este capítulo apresentou...


