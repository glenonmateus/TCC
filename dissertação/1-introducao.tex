\chapter{Introdução} \label{ch:introdução}
%- Introdução: Texto introdutório do trabalho, motivação, objetivos,
%metodologia, trabalhos relacionados.
%- Em cada capitulo adicionar um texto introdutório
%- Adicionar o seguinte texto:
%Este capítulo esta organizado da seguinte forma: A próxima seção
%apresenta .... A Seção XY apresenta...
Este trabalho apresenta um comparativo entre sistemas de detecção e prevenção de intrusão (Intrusion Detection and Prevention System - IDPS) de código aberto mais populares na comunidade de segurança da informação. Essas ferramentas permitem monitorar sistemas computacionais, alertando os administradores sobre possíveis ameaças.

Neste capítulo, apresentam-se, na \autoref{sec:motivação}, as motivações deste trabalho, evidenciando a importância do IDPS em um ambiente corporativo real. Em seguida, no \autoref{sec:objetivos}, os objetivos do trabalho, na \autoref{sec:metodologia}, uma descrição das metodologias utilizadas, na \autoref{sec:trabalhos-relacionados} os trabalhos relacionados, e por fim, na \autoref{sec:organização-do-trabalho}, a organização do trabalho.

\section{Motivação} \label{sec:motivação} 

A todo momento, novos ataques ou mesmo variações de ataques já existentes surgem e são lançados a várias redes indiscriminadamente em busca de vulnerabilidades. Em 2015, foram reportados ao Centro de Estudos, Respostas e Tratamento de Incidentes de Segurança no Brasil (CERT.br) cerca de 722.205 incidentes, em 2016, esse número diminuiu, chegando a 647.112 \cite{estatistica:cert.br}. Apesar da diminuição, esse número é considerado grande, presumindo-se que há muitos incidentes que não são reportados e/ou identificados.

Diante desse cenário, torna-se cada vez mais importante para o administrador de rede e/ou segurança da informação o uso de ferramentas de IDPS, permitindo identificar e tratar de forma automatizada os incidentes de segurança.

Para implementa tais ferramentas em uma rede, deve-se levar em consideração a flexibilidade e a administração simplificada para não resultar que a empresa/instituição fique na dependência de um único fabricante ou fornecedor da solução. Além disso, a ferramenta deve ter um bom desempenho e eficiência para não passar a falsa sensação de proteção ou degradar o desempenho da rede.

\section{Objetivos} \label{sec:objetivos}

Diante do apresentado, este trabalho tem como objetivo geral, analisar e apresentar um comparativo entre as soluções de código aberto de IDPS mais conhecidas: Snort e Suricata. Como desdobramento de tal objetivo, os seguintes objetivos secundários foram definidos:

\begin{alineas}
\item Descrever problemas relacionados a ataques envolvendo redes de computadores;
\item Descrever as ferramentas, compreendendo requisitos, características, modos de atuação e funcionalidades;
\item Descrever o ambiente experimental;
\item Realizar experimentos e coletar dados para validar o funcionamento e a eficiência das ferramentas;
\end{alineas}

\section{Metodologia} \label{sec:metodologia}
\section{Trabalhos Relacionados} \label{sec:trabalhos-relacionados}
\section{Organização do Trabalho} \label{sec:organização-do-trabalho}

Além deste capítulo introdutório, o restante do trabalho está divido da seguinte forma:

No \autoref{ch:segurança}, são definidos conceitos sobre redes de computadores e segurança da informação, também são descritos os ataques comuns e ferramentas utilizadas para validar e avaliar as soluções de IDPS.

Em seguida, no \autoref{ch:idps}, a definição IDPS, os tipos existentes, as funcionalidades e descrição das ferramentas avaliadas: Snort e Suricata.

O \autoref{ch:cenário-real} detalhará o cenário real e a infraestrutura utilizada para os realização dos testes das ferramentas, os testes realizados e o resultados obtidos.

Por fim, no \autoref{ch:considerações}, as considerações finais e trabalhos futuros.
