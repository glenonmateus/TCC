\chapter{Sistemas de Detecção e Prevenção de Intrusão} \label{ch:idps}

Os sistemas de detecção e prevenção de intrusão (\textit{Intrusion Detection and Prevention System} - IDS/IPS) são ferramentas de importância reconhecida pela comunidade da segurança da informação. Nesse capítulo, vamos apresentar os principais conceitos relacionados a IDS e IPS, uma breve descrição do funcionamento e classificação, para melhor entendimento das ferramentas que iremos apresentar e avaliar em um ambiente de real.

\section{Definições de IDS/IPS} \label{sec:ipds-definicoes}

\textit{Intrusion Detection Systems} (IDS) ou Sistemas de Detecção de Intrusão (SDI) são ferramentas utilizadas para monitoramento de eventos que ocorrem em redes e sistemas computacionais, analisando sinais de possíveis ataques que podem levar a uma violação das politicas de segurança da organização, alertando os administradores do sistema que estes eventos estão ocorrendo. 

O \textit{Intrusion Detection Systems} (IPS) ou Sistema de Prevenção de Intrusão (SPI) possui todas as funcionalidades do IDS com uma diferença, ele é capaz de deter os incidentes, minimizando os impactos causados por sistemas comprometidos \cite{mukhopadhyay01}.

%localizar a referencia
Os IDS's são compostos basicamente por quatro componentes, temos: 
\begin{alineas}
\item \textbf{Sensor ou Agente}: responsável pelo monitoramento e analise do trafego capturado; 
\item \textbf{Base de Dados}: usado como repositório das informações de eventos detectados pelo sensor e que posteriormente serão processados;
\item \textbf{Gestor}: é o dispositivo central que recebe, analisa e gerencia as informações de eventos vindo do sensor; 
\item \textbf{Console}: é uma interface para administração e monitoramento das atividades.
\end{alineas}

\section{Tipos de Sistemas de Detecção e Prevenção de Intrusão} \label{sec:idps-tipos}

Os IDPS's são classificados de acordo com o local onde o sensor é instalado, \textit{Host Based Intrusion Detection Systems} (HIDS) e \textit{Network Based Intrusion Detection Systems} (NIDS), e a técnica utilizada para o monitoramento, baseado em assinaturas e anomalias \cite{nagahama2012ipsflow}.

\subsection{Sistemas de Detecção de Intrusão Baseados em Host (HIDS)}

Em um HIDS o sensor é instalado no \textit{host}, monitorando as informações contidas na própria máquina. Esse tipo de IDS não observa o tráfego que passa pela rede (somente o trafego que passa pela placa de rede do \textit{host}), seu uso volta-se a verificação de informações relativas aos eventos e registros de logs e sistemas de arquivos (permissão, alteração, acesso a arquivos não autorizados) \cite{nagahama2012ipsflow}.  

As vantagens do HIDS são: 

\begin{alineas}
\item Evita a execução de códigos maliciosos;
\item Bloqueia tráfego de entrada e saída contendo ataques e uso não autorizado de protocolos e programas;
\item Evita que arquivos possam ser acessados, modificados e deletados impedindo a instalação de \textit{malwares} e ataques envolvendo acesso inapropriado a arquivos;
\end{alineas}

Por outro lado, o HID possui alguns desvantagens como \cite{scarfone01}:  

\begin{alineas}
\item Difícil instalação e manutenção;
\item Interfere no desempenho do \textit{hosts};
\item Demora para identificar eventos consequentemente a resposta ao incidente terá um atraso.
\end{alineas}

\subsection{Sistemas de Detecção de Intrusão Baseados em Rede (NIDS)}

No NIDS, o sensor é instalado na rede e a interface de rede atua em um modo especial chamado ``promíscuo'', tendo a capacidade de capturar o tráfego mesmo que os pacotes não sejam destinados ao sensor. Dessa forma, o NIDS monitora e analisa todo o trafego no segmento da rede, detectando atividades maliciosas, como ataques baseados em serviço, \textit{portscans}, entre outros, além de detectar se algum usuário legítimo está fazendo mau uso da rede \cite{nagahama2012ipsflow}.

Quanto a localização o NIDS pode ser classificado como passivo ou ativo. No modo passivo (\autoref{fig_nids-ativo}), o IDS monitora copias dos pacotes da rede que passam pelo \textit{switch} ou \textit{hub} onde está conectado, ficando limitado somente a gerar notificações quando encontrado algum tráfego malicioso. 

No entanto, no modo ativo (\autoref{fig_nids-passivo}), o IDS é instalado da forma que o tráfego da rede passe através do sensor parecendo com o fluxo de dados associado com um \textit{firewall}. Dessa forma, ele é capaz de parar ataques bloqueando o fluxo malicioso. 

É necessário uma analise minuciosa na instalação de um IDS ativo pois um mal dimensionamento de \textit{hardware} pode degradar a rede, adicionando atrasos excessivos aos pacotes.

As principais vantagens do um NIDS são: 

\begin{alineas}
\item São independentes de plataformas;
\item Não interfere no desempenho do \textit{host};
\item Fácil implantação e transparente para o atacante.
\end{alineas}

Dentre as desvantagens, temos:

\begin{alineas}
\item Pode adicionar retardados aos pacotes quando instalado no modo ativo;
\item Dificuldade de tratar dados de redes de alta velocidade;
\item Trata apenas segmentos de rede;
\item Dificuldade de tratar dados criptografados.
\end{alineas}

%Devido a grande heterogeneidade de dispositivos e sistemas operacionais na rede, a utilização desse tipo de IDS torna a administração mais simples se comparados com o HIDS.

\begin{figure}[htb]
 \label{fig_nids-arquitetura}
 \centering
 \begin{minipage}{0.4\textwidth}
  \centering
  \caption{Exemplo de arquitetura de NIDS passivo} \label{fig_nids-passivo}
  \includegraphics[scale=.55]{nids_passivo.png}
  \legend{Fonte: Autoria própria}
 \end{minipage}
 \hfill
 \begin{minipage}{0.4\textwidth}
  \centering
  \caption{Exemplo de Arquitetura de NIDS ativo} \label{fig_nids-ativo}
  \includegraphics[scale=.55]{nids_ativo.png}
  \legend{Fonte: Autoria própria}
 \end{minipage}
\end{figure}

\subsection{Sistema de Detecção de Intrusão Distribuídos}

A função de um Sistema de Detecção de Intrusão Distribuído (SDID) é de gerencia. Os sensores (pode ser NIDS, HIDS ou a combinação de ambos), localizados remotamente, reportam os alertas para um centralizador. Os \textit{logs} de ataques são, periodicamente, enviados para a estação de gerenciamento, armazenando em uma base única e centralizada, além disso, novas assinaturas de ataques podem ser enviadas para os sensores \cite{snort:andrew}.   

\begin{figure}[!htb]
  \centering
  \caption{Sistema de Detecção de Intrusão Distribuído} \label{fig_dids}
  \includegraphics[scale=0.7]{dids.png}
  \legend{Fonte: Autoria própria}
\end{figure}

Na \autoref{fig_dids} mostra um SDID composto por dois sensores e um estação de gerenciamento centralizado. O sensor NIDS 1 e NIDS 2 estão operando em modo \textit{promiscuos} e está protegendo segmentos de rede. É recomendando que a conexão entre os sensores e o centralizado seja feita por uma rede privada, em caso de utilização de rede públicas, recomenda-se adicionar uma camada de segurança, como criptografia, ou VPN.

\subsection{Formas de Detecção} \label{sec:idps-formas}

Quanto a técnica de monitoramento utilizado, o IDS pode ser baseados em assinaturas ou anomalias. IDSs baseados em assinaturas compara os pacotes com uma base de assinaturas de ataques previamente conhecidos e reportados por especialistas, cada assinatura identifica um ataque \cite{nagahama2012ipsflow}.  

As vantagens de um IDS baseados em assinaturas são:

\begin{alineas}
\item Usa pouco recurso de \textif{hardware} do servidor;
\item Possui, de certa forma, um rápido processamento.
\end{alineas}

Dentre as desvantagens temos:

\begin{alineas}
\item Exige uma atualização constante da base de assinaturas;
\item Para a geração de uma base própria, a equipe precisa de um alto conhecimento técnico;
\item Possui altos índices de falsos positivos e negativos.
\end{alineas}

Os IDS baseados em anomalias, procuram determinar um comportamento normal na fase de aprendizagem do sistema computacional ou rede e sempre que existir um desvio desse padrão alertas são gerados. 

Possui a vantagem de detectar novos ataques sem necessariamente conhecer a fundo a intrusão através dos desvios de comportamento. Porém, tem como desvantagem a geração de um grande número de falsos alertas em decorrência a modificações na rede ou \textit{host} nem sempre representar um tráfego malicioso. 

\section{Principais Ferramentas de IDS} \label{sec:idps-ferramentas}

Nesse capitulo, será apresentado as ferramentas de IDPS analisadas. A escolha dessas ferramentas deu-se devido ser de código aberto e de livre uso, e também, pela sua popularidade diante da comunidade de segurança da informação.

\subsection{Snort} \label{sec:snort}

O Snort é um sistema de detecção e prevenção de intrusão de código fonte aberto escrita na linguagem de programação C bem conhecido pela comunidade da segurança da informação. Seu primeiro \textit{release} foi lançado em 1998 e desde então passa por constantes revisões e aperfeiçoamentos, com o passar dos anos se tornou o IDS mais utilizado no mundo. Ele combina análise baseada em assinaturas e anomalias, podendo operar em três modos: \textit{sniffer}, \textit{packet logger} e de sistema de detecção de intrusão (NIDS) \cite{snort:manual}.

No modo \textit{Sniffer}, o Snort captura os pacotes e exibi as informações no console de forma continua. No modo \textit{Packet Logger}, além de capturar o tráfego, o Snort escreve essas informações em arquivos (chamados de logs) que são armazenados no disco. Por fim, o \textit{Network Intrusion Detection System} - NIDS, sendo o modo mais complexo e completo, permitindo capturar e analisar os pacotes de rede em tempo real \cite{snort:manual}.

Existe quatro componentes no Snort: O \textit{sniffer}, o pré-processador, o motor de detecção e módulos de saída. A \autoref{fig_snort-componentes} mostra a arquitetura e disposição dos componentes \cite{snort:andrew}.

\begin{figure}[!htb]
  \centering
  \caption{Arquitetura do Snort} \label{fig_snort-componentes}
  \includegraphics[scale=0.6]{snort_componentes}
  \legend{Fonte: Autoria própria}
\end{figure}

O pré-processador, o motor de detecção e os componentes de alerta do Snort são todos \textit{plugins}. Os \textit{Plugins} são programas escritos em conformidade com a API de \textit{plugins} do Snort. Esses programas são usados no core do Snort, mas eles são separados para que as modificações feitas no \textit{core} sejam mais confiáveis e mais fáceis de realizar \cite{snort:andrew}.

O \textit{sniffer} é um dispositivo (\textit{software} ou \textit{hardware}) usado para ver o trafego passante em algum segmento de rede. No caso da Internet, consiste geralmente de trafico IP (composto por diferentes protocolos de alto nível como, TCP, UDP, ICMP, protocolos de roteamento e IPSec). Os pacotes são analisado, interpretados e exibidos de uma forma legível para os humanos.

Um \textit{sniffer} tem os seguintes usos:

\begin{alineas}
\item Analisador de rede e resolução de problemas;
\item Analisador de performance e avaliação comparativa;
\item Capturar senhas em texto plano e outros dados sensíveis.
\end{alineas}

Assim como qualquer outra ferramenta de rede, os \textit{sniffers} podem ser usados tanto para o bem quanto para o mal. Então, criptografar o trafego de rede previne que pessoas sejam capazes de lerem os pacotes capturados \cite{snort:andrew}.

O pré-processador pega o pacote bruto e faz uma checagem utilizando um determinado \textit{plugin}. Esses \textit{plugins} verificam se o pacote tem um tipo particular de comportamento, uma vez determinado, o pacote é enviado para o motor de detecção caso contrário é descartado.

Na \autoref{fig_snort_preprocessor}, pode-se ver como o pré-processador utiliza \textit{plugins} para chegar pacotes. O Snort suporta muitos tipos de pré-processadores, cobrindo vários protocolos comumente usados como, IP \textit{fragmentation handling}, \textit{port scanning} e controle de fluxo.

\begin{figure}[!htb]
  \centering
  \caption{Uso de \textit{plugins} no pré-processador} \label{fig_snort_preprocessor}
  \includegraphics[scale=0.8]{snort_preprocessor.png}
  \legend{Fonte: Autoria própria}
\end{figure}

O uso de \textit{plugins} é uma característica muito útil para o IDS, pois os \textit{plugins} podem ser ativados e desativadas a medida do necessário, otimizando a utilização dos recursos computacionais e geração de alertas \cite{snort:andrew}.

Os pacotes, após passarem por todos os pré-processador, são entregues para o motor de detecção. O motor de detecção pega esses dados e faz uma checagem utilizando uma base de regras pré-configurado pelo administrador. Se a regra for compatível com os dados do pacote, eles são enviado para o processador de alertas, caso contrário, são descartados \cite{snort:andrew}.

Na \autoref{fig_snort_detecção}, temos os pacotes saindo dos pré-processadores e chegando no motor de detecção. No motor de detecção há uma base de regras configurada, os dados dos pacotes são comparados com as assinaturas da base, se coincidirem, uma ação é tomada, caso contrário, o pacote é descartado.

\begin{figure}[!htb]
  \centering
  \caption{Motor de Detecção do Snort} \label{fig_snort_detecção}
  \includegraphics[scale=0.8]{snort_detection.png}
  \legend{Fonte: Autoria própria}
\end{figure}

A base de regras é um conjunto de assinaturas de ataques conhecidos e catalogados. As regras são escritas em formato texto em uma única linha e constituídas por duas partes: 

\begin{alineas}
\item \textbf{Cabeçalho}: São definidos que ações serão tomadas, tipo de pacote (TCP, UDP, ICMP, etc), o IP de origem e destino e porta;
\item \textbf{Opções}: É o conteúdo do pacote que faz ele ser compatível com a regra.
\end{alineas}

Dentre as ações que podem ser tomadas temos:

\begin{alineas}
\item \textbf{\textit{Activation}}: Alerta e chama regra do tipo \textit{dynamic};
\item \textbf{\textit{Dynamic}}: permanece inativa até ser ativado por uma regra \textit{activate}, registrando o tráfego;
\item \textbf{\textit{Alert}}: Gera um alerta usando um método selecionado e então registra os pacotes e dados;
\item \textbf{\textit{Pass}}: Ignora os pacotes;
\item \textbf{\textit{Log}}: Registra e não alerta.
\end{alineas}

Abaixo temos um exemplo de regra.

\begin{lstlisting}
alert icmp any any -> any any (msg:"Ping suspeito"; 
sid:1; resp:icmp_all;)
\end{lstlisting}

Com a regra acima, o Snort gerará um alerta de qualquer pacotes ICMP que estiver passando de qualquer máquina e porta origem (\textbf{any any}) para qualquer máquina e porta destino (\textbf{any any}) e enviará pacotes ICMP para a máquina de origem com as mensagens \textit{host unreachable; network unreachable}.

\subsection{Suricata} \label{sec:suricata}
\section{Conclusão} \label{sec:idps-conclusao}
%Ex: Este capítulo apresentou...
