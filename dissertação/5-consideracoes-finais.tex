\chapter{Considerações Finais e Trabalhos Futuros} \label{ch:considerações}

Diante dos resultados obtidos na \autoref{sec:resultados}, conclui-se que, apesar da ferramenta Suricata (versão 4.0.0) ter, em aspectos gerais, um desempenho melhor, justificando assim, que a arquitetura nela implementada (utilizando \textit{multithread}) ajuda na melhoria da detecção e geração de alertas. 

No entanto, recomenda-se a utilização do Snort (versão 2.9.8.3) em um ambiente de produção real. Tal afirmação, é sustentada devido a um comportamento do Suricata observado durante o desenvolvimento desse trabalho, que, por algum motivo não investigado, era finalizado. Dessa forma, os alertas não eram gerados, deixando a rede um pouco desprotegida.

Além disso, caso opte-se por implantar o Suricata em um ambiente de produção, será necessário um esforço maior, pois a equipe de segurança, deve está sempre monitorando se o processo da ferramenta está executando e gerando os alertas corretamente.

A realização desse trabalho foi de grande valor para aquisição de conhecimento na área de segurança de redes. Concluindo-se que, embora haja na rede um IDS, somente a implementação deste não é suficiente para se ter uma segurança total, e sim, que ele deva atuar em paralelo com outros sistemas de segurança, vindo assim a somar na dura batalha pela obtenção de segurança em redes. 


