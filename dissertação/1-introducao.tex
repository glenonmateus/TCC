\chapter{Introdução} \label{ch:introdução}
%- Introdução: Texto introdutório do trabalho, motivação, objetivos,
%metodologia, trabalhos relacionados.
%- Em cada capitulo adicionar um texto introdutório
%- Adicionar o seguinte texto:
%Este capítulo esta organizado da seguinte forma: A próxima seção
%apresenta .... A Seção XY apresenta...
Este trabalho apresenta um comparativo entre sistemas de detecção e prevenção de intrusão (Intrusion Detection and Prevention System - IDPS) de código aberto mais populares na comunidade de segurança da informação. Essas ferramentas permitem monitorar sistemas computacionais, alertando os administradores sobre possíveis ameaças.

Neste capítulo, apresentam-se, na \autoref{sec:motivação}, as motivações deste trabalho, evidenciando a importância do IDPS em um ambiente corporativo real. Em seguida, no \autoref{sec:objetivos}, os objetivos do trabalho, na \autoref{sec:metodologia}, uma descrição das metodologias utilizadas, e por último, na \autoref{sec:relacionados} os trabalhos relacionados.

\section{Motivação} \label{sec:motivação} 

A todo momento, novos ataques ou mesmo variações de ataques já existentes surgem e são lançados a várias redes indiscriminadamente em busca de vulnerabilidades. Em 2015, foram reportados ao Centro de Estudos, Respostas e Tratamento de Incidentes de Segurança no Brasil (CERT.br) cerca de 722.205 incidentes, em 2016, esse número diminuiu, chegando a 647.112 \cite{estatistica:cert.br}. Apesar da diminuição, esse número é considerado grande, presumindo-se que há muitos incidentes que não são reportados e/ou identificados.

Diante desse cenário, torna-se cada vez mais importante para o administrador da rede o uso de ferramentas de IDPS, permitindo identificar e tratar de forma automatizada os incidentes de segurança.

\section{Objetivos} \label{sec:objetivos}
\section{Metodologia} \label{sec:metodologia}
\section{Trabalhos Relacionados} \label{sec:relacionados}
