\chapter{Segurança em Redes de Computadores} \label{ch:segurança}

Esse capítulo apresentará conceitos e definições sobre segurança da informação e rede de computadores, mostrando seus principais componentes e os ataques mais utilizados contra essas redes. Ao final, foram apresentadas as ferramentas usadas para simular ataques com o intuito de avaliar o comportamento dos IDPS (Snort e Suricata).

Este capítulo esta organizado da seguinte forma: A próxima seção apresenta as definições sobre segurança da informação. Na \autoref{sec:cenario-geral} será apresentado uma topologia de rede comum, que existe na organizações. Na \autoref{sec:pontos-vulnerabilidade} será abordado os pontos vulneráveis em uma rede. Na \autoref{sec:ataques-comuns} será apresentado o principais ataques a rede de computadores. Por fim, na \autoref{sec:ferramentas} será apresentada as ferramentas usadas para gerar os ataques.

\section{Definições} \label{sec:definições}

Quando se fala e segurança de sistemas computacionais, logo vem a mente da maioria dos usuários da rede, roubo de número de cartões de crédito, \textit{hackers} danificando páginas (\autoref{sec:desfiguração}) e aplicações \textit{Web} e ataques de negação de serviço (\autoref{sec:negação}). Também tens a imagem dos \textit{malwares}, como vírus, cavalos de tróia e \textit{worms} (\autoref{sec:malwares}). Esses possuem maior visibilidade pois representam uma parte significativa das ameaças existentes na Internet.

Porém existem outros problemas que apresentam riscos que normalmente não são levados em consideração, como administradores desonestos, funcionários descontentes e usuários que utilizam dados sigilosos de forma equivocada.

Para um melhor entendimento sobre segurança da informação, precisa-se entender alguns elementos listados abaixo \cite{esr-gestao}:

\begin{alineas}
\item \textbf{Incidente de segurança}: qualquer evento oposto a segurança; por exemplo, ataques de negação de serviços (\textit{Denial of Service} - DoS), roubo de informações, vazamento e obtenção de acesso não autorizado a informações;
 \item \textbf{Ativo}: qualquer coisa que tenha valor para a organização e para seus negócios. Alguns exemplo: banco de dados, softwares, equipamentos (computadores e notebooks), servidores, elementos de redes (roteadores, switches, entre outros), pessoas, processos e serviços;
 \item \textbf{Ameaça}: qualquer evento que explore vulnerabilidades. Causa potencial de um incidente indesejado, que pode resultar em dano para um sistema ou organização;
 \item \textbf{Vulnerabilidade}: qualquer fraqueza que possa ser explorada e comprometer a segurança de sistemas ou informações. Fragilidade de um ativo ou grupo de ativos que pode ser explorada por uma ou mais ameaças. Vulnerabilidades são falhas que permitem o surgimento de deficiências na segurança geral do computador ou da rede. Configurações incorretas no computador ou na segurança também permitem a criação de vulnerabilidades. A partir dessa falha, as ameaças exploram as vulnerabilidades, que, quando concretizadas, resultam em danos para o computador, para a organização ou para os dados pessoais;
 \item \textbf{Risco}: probabilidade de uma ameaça se concretizar;
 \item \textbf{Ataque}: qualquer ação que comprometa a segurança de uma organização;
 \item \textbf{Impacto}: consequência de um evento.
\end{alineas}

Diante desses elementos, podemos definir segurança da informação como sendo a proteção das informações, sistemas, recursos e demais ativos contra desastres, erros (intencionais ou não) e manipulação não autorizada, objetivando a redução da probabilidade e do impacto de incidentes de segurança.

Segundo \cite{isoiec27002}, segurança da informação é a preservação da confidencialidade, da integridade e da disponibilidade da informação, adicionalmente, outras propriedades, tais como autenticidade, responsabilidade, não repúdio e confiabilidade, podem também estar envolvidas.

Dentre vários conhecimentos que um profissional de segurança deve possuir, o conceito mais básico e considerado o pilar de toda a área de segurança corresponde à sigla CID (Confidencialidade, Integridade e Disponibilidade), de modo que um incidente de segurança é caracterizado quando uma dessas áreas é afetada \cite{seg-redes-sistemas}. Abaixo será detalhado cada item.

\begin{alineas}
 \item \textbf{Confidencialidade}: termo ligado à privacidade de um ativo ou recurso, que deve ser acessível somente por pessoas ou grupos autorizados;
 \item \textbf{Integridade}: possui duas definições, a primeira está relacionada com o fato da informação ter valor correto, a segunda, está ligada à inviolabilidade da informação;
 \item \textbf{Disponibilidade}: está relacionada ao acesso à informação, que deve está disponível quando necessária.
\end{alineas}

Dois dos termos citados são fáceis de ser monitorados pois é perceptível para o usuário: a integridade (capacidade de identificar se uma informação foi alterada) e a disponibilidade (através da tentativa de acesso a um serviço e verificando se o mesmo está respondendo adequadamente). No entanto, só é possível identificar se houve quebra da confidencialidade com auditorias, analisando os registros de acesso (se houver), isso torna a identificação custosa e em muitos casos impossível \cite{seg-redes-sistemas}.

Além dos conceito listados, a literatura moderna considera mais alguns conceitos auxiliares, temos:

\begin{alineas}
 \item \textbf{Autenticidade}: garantia que uma informação, produto ou documento foi elaborado ou distribuído pelo autor a quem se atribui;
 \item \textbf{Legalidade}: garantia de que ações sejam realizadas em conformidade com os preceitos legais vigentes e que seus produtos tenham validade jurídica;
 \item \textbf{Não repúdio}: conceito bastante utilizado em certificação digital, onde o emissor de uma mensagem não pode negar que a enviou;
 \item \textbf{Privacidade}: habilidade de uma pessoa controlar a exposição e a disponibilidade de informações acerca de si.
\end{alineas}

Os ataques são classificados em passivo e ativo. Em um ataque passivo não há interação direta (modificações de arquivos ou afetando os recursos) com o sistema alvo, o atacante apenas monitora com o objetivo de obter informações. Por outro lado, os ataques ativos há modificações de dados que afetam as operações do sistema. Os ataques podem ser divididos em categorias apresentadas na tabela.

\begin{table}[htb]
\ABNTEXfontereduzida
\centering
\caption{Classificação dos ataques passivos e ativos}
\label{tab:tipos-de-ataques}
\begin{tabular}{|M{2cm}|M{5cm}|M{6cm}|}
    \hline
    \textbf{Ataque} & \textbf{Categoria} & \textbf{Descrição} \\
    \hline
    \multirow{2}{*}{Passivo} & Liberação de conteúdo da mensagem & Ocorre quando uma informação é captada e seu conteúdo é lido pelo atacante \\
    \cline{2-3}
    & Análise de tráfego & Ocorre quando o tráfego da troca de uma informação (criptografada ou não) é analisado para identificar padrões nas mensagens \\
    \hline
    \multirow{4}{*}{Ativo} & Disfarce & Ocorre quando uma entidade finge ser outra entidade \\
    \cline{2-3}
          & Repetição & Ocorre quando os dados são capturados passivamente e, subsequentemente, retransmitidos para produzir um efeito não autorizado \\
    \cline{2-3}
          & Modificação da mensagem & Ocorre quando alguma parte da mensagem original é alterada para produzir um efeito não autorizado \\
    \cline{2-3}
          & Negação de serviço & Ocorre quando há um impedimento ou inibição do uso ou gerenciamento normal das instalações de comunicação \\
    \hline
\end{tabular}
\legend{Fonte: Autoria própria} 
\end{table}

Além disso, podemos dividir os ataques em quatro categorias, que são \cite{sistemasids:joao}: 

\begin{alineas}
\item \textbf{Interrupção}: Esse ataque tem como objetivo interromper ou destruir o serviço, afetando a disponibilidade da informação, como ocorre, por exemplo, nos ataques de negação de serviço (DoS) e ataques de negação de serviço distribuído (DDoS);
\item \textbf{Interceptação}: Esse ataque visa capturar informações que estão em transito sem a percepção do vítima, comprometendo sua privacidade. Seu objetivo principal é gerar cópias de informações, arquivos e programas de forma não autorizada. Um exemplo desse tipo de ataque é o \textit{Man-in-the-Middle}.
\item \textbf{Modificação}: Esse ataque ocorre quando as informações transmitidas são alteradas, após serem captadas, afetando sua integridade. Como exemplo desse ataque temos o \textit{Replay Attack}.
\item \textbf{Falsificação}: Esse ataque tem como finalidade se passar por um usuário do sistema para obter informações e transmiti-las na rede, comprometendo a autenticidade da informação. Como exemplo desse ataque temos o IP \textit{Spoofing}.
\end{alineas}

\section{Cenário Geral} \label{sec:cenario-geral}

Nessa seção será explicado alguns conceito básico sobre redes de computadores e descrever uma topologia de rede genérica conectada a internet.

Uma rede de computadores é um conjunto de dispositivos interconectados para compartilhar recursos como \textit{hardware}, \textit{software}, interação e interatividade, onde existem máquinas que desempenham os papéis de clientes, servidores e/ou parceiros dependendo do serviços disponíveis na rede \cite{modelo:jose}.

As características de uma rede são: dois ou mais computadores interligados; meio físico de comunicação (com fio, sem fio, metálico, fibra, etc); vários tipos de equipamentos (estações de usuários, servidores, concentradores, etc); software para comunicação entre os equipamentos (protocolos); aplicativos para transferência de informação \cite{esr:arquitetura}.

As redes podem ser classificadas em duas categorias \cite{esr:arquitetura}: 
\begin{alineas}
\item \textbf{Redes par-a-par ou peer-to-peer}: Aqui não existem servidores dedicados ou hierarquia entre os computadores, todos são iguais, onde cada computador funciona como cliente e/ou servidor, cabendo o usuário determinar o que será compartilhado (\autoref{fig_p2p});
\item \textbf{Redes cliente-servidor}: Aqui há servidores dedicados que oferecem serviços à rede (servidores de arquivos e impressão, correio, fax, comunicação, aplicações, etc), geralmente são otimizadas para processar rapidamente as requisições dos clientes da rede e para garantir a segurança dos arquivos e pastas (\autoref{fig_cliente-servidor}). 
\end{alineas}

\begin{figure}[!htb]
 \label{fig:arquitetura-redes}
 \centering
 \begin{minipage}{0.4\textwidth}
  \centering
  \caption{Rede par-a-par} \label{fig_p2p}
  \includegraphics[scale=.4]{p2p.png}
  \legend{Fonte: Autoria própria}
 \end{minipage}
 \hfill
 \begin{minipage}{0.4\textwidth}
  \centering
  \caption{Rede cliente-servidor} \label{fig_cliente-servidor}
  \includegraphics[scale=.4]{cliente-servidor.png}
  \legend{Fonte: Autoria própria}
 \end{minipage}
\end{figure}

Uma Internet é uma rede de computadores que interconecta centenas de dispositivos de computadores ao redor do mundo \cite{redes:kurose}. A União Internacional de Telecomunicações (UTI) estima que haja cerca de 3.578 milhões (\autoref{fig:estatisca-itu}) de usuários usando diferentes tipos de dispositivos, como, celulares, automóveis, \textit{webcams}, TVs, \textit{laptops}, consoles para jogos, entre outros \cite{estatistica:itu}.

\begin{figure}[htb]
    \centering
    \caption{Quantidade de usuário conectados na Internet} 
    \includegraphics[scale=.55]{estatistica-uit.png}
    \legend{Fonte: \cite{estatistica:itu}}
    \label{fig:estatisca-itu}
\end{figure}

Os equipamentos que são comumente usados nas redes de computadores são:

\begin{alineas}
\item \textbf{Concentradores (\textit{hubs})}: São pontos de conexões para dispositivos em uma rede, contendo várias portas usados para conectar os segmentos da LAN. Quando um pacote chega em uma porta, ele é replicado para as demais portas, assim, todos os clientes conectados ao \textit{hub} podem ver todos os pacotes. Esse tipo de equipamento não é mais recomendado.
\item \textbf{Switches}: São equipamentos que se diferem dos \textit{hubs} por serem capazes de ler o MAC de origem e destino. Além disso, realizam comutações (os pacotes são individualmente encaminhadas entre os dispositivos conectados) de quadros na camada de enlace;
\item \textbf{Roteadores}: São dispositivos de rede mais tradicionais, como de backbone das intranets e da internet. Suas principais funções são seleção dos melhores caminhos de saída para os pacotes de entrada e roteamento destes pacotes para a interface de saída apropriada.
\end{alineas}

Um administrador, com o minimo de consciência sobre segurança, coloca em sua rede, um \textit{firewall} de borda. Um \textit{firewall} sempre é colocada na divisa entre duas ou mais redes, pode ser entre redes privadas ou entre uma rede privada e a Internet. Uma empresa pode ter muitas LANs conectadas de forma arbitrárias, mas todo o tráfego de saída ou de entrada da empresas para a internet deve ser feito através do \textit{firewall}, permitindo assim, que alguns pacotes passem e bloqueando outros \cite{redesdecomputadores}.

Há três tipos básicos de \textit{firewall}, os mais tradicionais são os filtros de pacotes e os \textit{proxies}. O terceiro tipo é uma evolução do filtro de pacotes tradicional chamado de filtro de estados de pacotes ou \textit{stateful packet filter} (SPF) \cite{univhacker}.

Um \textit{firewalls} de filtros de pacotes são baseados em tabelas configuradas pelo administrador da rede. Essas tabelas listam as origens e os destinos aceitáveis e/ou bloqueados e as regras padrões que orientam o que deve ser feito com os pacotes recebidos de outras máquinas ou destinados a elas, ou seja, o \textit{firewall} tem como função controlar o tráfego entre as redes \cite{redesdecomputadores}. 

Há vários \textit{softwares} que implementam filtro de pacotes. Alguns são instalados em \textit{hardwares} como roteadores outros são programas que rodam em computadores comuns \cite{univhacker}. Um utilitário bastante conhecido e utilizado para essa finalidade é o \textit{iptables}. A \autoref{tab:firewall-regras} apresenta um exemplo de uma tabela de regras.

\begin{table}[htb]
\ABNTEXfontereduzida
\centering
\caption{Tabela de regras aplicadas no \textit{firewall}}
\label{tab:firewall-regras}
\begin{tabular}{l|l|l|l|l|l|l}
    \textbf{IP Origem} & \textbf{IP Destino}  & \textbf{Porta Origem}  & \textbf{Porta Destino} & \textbf{Protocolo} & \textbf{Flag TCP} & \textbf{Ação} \\ \hline
    Rede Externa & Servidor Web & Todas & 80,443 & TCP & Todos & Permitir \\ \hline
    Rede Externa & Servidor Web & Todas & 21,3000:3070 & TCP & Todos & Permitir \\ \hline
    Todas & Todas & Todas & Todas & Todos & Todos & Negar \\
\end{tabular}
\legend{Fonte: Autoria própria} 
\end{table}

No exemplo, são permitidas conexões na Intranet no Servidor Web pelas portas 80 e 443 (padrão nos protocolos HTTP e HTTPS) para todos os \textit{Flags} TCP (ACK, ACK/SYN, SYN e FIN). Além disso, podemos definir um range de portas, como na linha 2, que são abertas as portas 21 e todas as portas entre 3000 e 3070, utilizadas por padrão pelo protocolo FTP.

Um \textit{proxie} trabalha na camada de aplicação interagindo com o programa e seus protocolos, independente de como esse protocolo será encapsulado na pilha TCP/IP. Por exemplo, um \textit{proxy} para Web trabalha apenas com o protocolo HTTP, bloqueando os demais. Além disso, pode-se configurar-lo para controlar quem pode ou não acessar serviços externos \cite{univhacker}.

No \textit{firewall} de filtros de pacotes por estado (SPF) uma nova tecnologia de análise de pacotes foi agregada, permitindo que eles lembrem-se de pacotes anteriores antes de permitir outro mais recente entrar. Isso é implementado na forma de uma tabela de conexões ativas. Quando uma conexão é iniciada, todos os dados do pacote são guardados nela. Se um novo pacote chegar em direção à mesma máquina, o SPF consulta a tabela. O novo pacote e aceito caso seja dada a continuação da conexão ou rejeitado, se não for \cite{univhacker}.

Na \autoref{fig:cenario-geral} apresenta uma típica rede composta por um roteador de núcleo que interliga roteadores (A, B e C) de outras redes da Intranet, que por sua vez interliga os clientes e/ou servidores. Todo trafego de saída e entrada da Intranet para a Internet passa pelo roteador de núcleo, além disso, os pacotes são tratados por um \textit{firewall} de borda, que determina o que entra e o que sai da rede local.

\begin{figure}[!htb]
    \centering
    \caption{Topologia geral de uma rede de computadores} 
    \includegraphics[scale=.6]{cenario-geral.png}
    \legend{Fonte: Autoria própria}
    \label{fig:cenario-geral}
\end{figure}

\section{Pontos de Vulnerabilidade} \label{sec:pontos-vulnerabilidade}
%Ex.: Roteador, firewall, clientes e suas aplicações, serviço mal configurado (sem aplicação de patch de atualização)
% falar dos problemas nas aplicações (erro de programação, sqlinjection, cross-site script)
% problemas relacionados a firewall que na sua maioria é fitro de pacotes, o atacante pode usar portas abertas no firewall que possuem serviços com falhas de segurança conhecidas ou aplicações mal desenvolvidas.
% clientes, envolve engenharia social, mencionar senha fraca, prática de spam

Nessa seção será abordado os pontos fracos que uma pessoa má intencionada pode explorar para ter um ataque bem sucedido a um rede de computador.

Apesar da preocupação dos administradores em proteger suas redes de ataques, devido a sua heterogeneidade, sempre haverá uma breja a ser explorada. A literatura considera o ser humano como elo mais fraco, é bastante comum o usuário cadastrar senhas fracas, por conveniência, e fácil memorização (\autoref{sec:forçabruta}).

Uma técnica bastante comum usadas por golpistas que visa obter informações financeiras ou informações pessoais da vítima, é o \textit{phishing}. Os ataques de \textit{phshing} mais conhecidos são os quais um atacante induz usuários a acessar um site clonado de uma instituição financeira, de modo a coletar suas credenciais de acesso \cite{esr:tratamento}. Muitos usuário, por falta de conhecimento ou até por ingenuidade, acabam informando seus dados em sites falsos, sem ao menos verificar a veracidade do mesmo.

Outros serviços visados por esse tipo de ataque são \cite{esr:tratamento}:
\begin{alineas}
\item Credenciais de serviços: e-mails, redes sociais ou armazenamento;
\item Webmail corporativo; 
\item Programa de milhagem (companhias aéreas ou redes de supermercados);
\item Comércio eletrônico.
\end{alineas}

Uma forma de difundir \textit{phishing} é através de \textit{spam}. Um \textit{spam} é uma mensagem, na maioria das vezes de conteúdo falso, enviado para diversos e-mails, nessas mensagens podem conter uma URL de um site falso ou um site com códigos maliciosos \autoref{sec:malwares} que infectam a vítima. Um \textit{spammer}, como é chamado quem envia \textit{spam}, também usa técnica de \textit{phishing} para obter dados de acesso de e-mails pessoais. Dessa forma, o \textit{spammer} pode acessar a conta de e-mail da vítima e enviar \textit{spam} para todos os seus contatos, sem que a vítima perceba.

\begin{figure}[htb]
 \centering
 \caption{Fraude identificada pelo CAIS}
 \includegraphics[scale=.6]{fraude-spam.png}
 \legend{Fonte: \cite{cais}}
 \label{fig:spam}
\end{figure}

O CAIS mantém um catálogo de fraudes identificadas sobre os principais golpes que estão em circulação. Na \autoref{fig:spam}, temos uma fraude recebida por e-mail (\textit{spam}) contendo um \textit{link} para download de um arquivo malicioso criado para roubar informações da vítima e instalar outros arquivos \cite{cais}.

Outro problema comum é o desenvolvimento de aplicações \textit{web} sem nenhuma preocupação com segurança, podendo comprometer, não somente o serviço, mas também, em casos mais extremos, o servidor inteiro. Para tal, atacante pode usar vários artifícios, os mais conhecidos são \textit{sql injection} e \textit{cross-site script}.

A inserção de \textit{Structured Query Language} (SQL) via formulário na aplicação \textit{web} resulta num ataque de \textit{sql injection}. O atacante injeta um código dentro dos campos de entrada, como usuário e senha, de uma aplicação onde a declaração condicional sempre será verdadeiro quando executado. Em casos bem sucedidos, o atacante pode alterar o banco de dados, acessar informações sensíveis ou ter acesso ao sistema \cite{sqlinjection:sankaran}.

No exemplo abaixo, a declaração condicional 'OR 1=1' torna toda a clausura WHERE verdadeiro pois a expressão 1=1 é uma tautologia. A consulta retorna todos os dados da tabela \textit{user\_info}. Perceba que os dois hífens fornecidos no final da entrada comenta o resto da linha.

\begin{lstlisting}[Language=SQL, frame=single] 
  SELECT * FROM user_info WHERE logID="" OR 1=1 -- AND pass1="" 
\end{lstlisting}

O \textit{Cross-site scripting} (XSS) é uma forma de ataque que permite utilizar um aplicação vulnerável para transporta códigos maliciosos até o navegador de outros usuário. O navegador da vítima entende que o código recebido é legítimo e, por isso, informações sensíveis, como o identificador de sessão do usuário, por exemplo, podem ser acessadas programaticamente \cite{pentestweb:nelson}.

Com o XSS pode-se roubar histórico de navegação, fazer uma varredura de redes privadas, descobrir consultas realizadas em mecanismos de busca, escravizar o navegador \textit{web} e proliferar \textit{worms} (\autoref{sec:malwares}) baseados em XSS \cite{pentestweb:nelson}. 

\section{Ataques Comuns à Redes de Computadores} \label{sec:ataques-comuns}

Nessa seção será descritos os ataques mais comuns à redes e serviços de organizações privadas e públicas, financeiras ou acadêmicas. Para licitar os ataques dessa seção, levou-se em consideração as estatísticas divulgada pelo CERT.br (\autoref{fig:cert}).

O CERT.br é o grupo de resposta a incidentes de segurança para a internet brasileira, mantido Comitê Gestor da Internet no Brasil. Atua na notificação e tratamento de incidentes de segurança dando apoio no processo de resposta. Além disso, faz um trabalho de conscientização e treinamento sobre problemas de segurança no Brasil. 

\begin{figure}[htb]
 \centering
 \caption{Estatísticas de ataques reportadas ao CERT.br}
 \includegraphics[scale=.6]{incidentes-reportados.png}
 \legend{Fonte: \cite{tipos-ataques:certs.br}}
 \label{fig:cert}
\end{figure}

Paralelamente ao CERT.br temos o Centro de Atendimento a Incidentes de Segurança (CAIS), mantido pela Rede Nacional de Ensino e Pesquisa (RNP). O CAIS é responsável por zela pela segurança da rede Ipê (infraestrutura de rede dedicada à comunidade brasileira de ensino superior), detectando, resolvendo e prevenindo incidentes de segurança. Além disso, tem o papel de orientar (através de publicações de cartilhas) e disseminar boas práticas de segurança da informação, educando e conscientizando usuários de todos os níveis sobre os principais riscos em segurança da informação \cite{cais}.

Desde 2008, todas a fraudes identificadas pelo CAIS estão sendo ordenadas e disponibilizadas para consulta (\autoref{fig:cais}). Adicionalmente, são enviados alertas através de uma lista quando uma fraude mostra-se particularmente perigosa aos usuários e computadores.

\begin{figure}[htb]
 \centering
 \caption{Estatísticas de incidentes reportados ao CAIS}
 \includegraphics[scale=.7]{cais-incidentes-reportados.png}
 \legend{Fonte: \cite{cais}}
 \label{fig:cais}
\end{figure}

\subsection{\textit{Scanners}} \label{sec:scanners}

Conforme tratado na \autoref{sec:definições}, uma vulnerabilidade é a fraqueza em sistemas de informação, procedimentos de segurança do sistema e controles internos, ou aplicação que pode ser explorada tendo como origem uma ameaça. 

\textit{Scanners} são programas usados para varrer uma rede à procura de computadores (tanto pessoais como servidores) com alguma vulnerabilidade. Podemos dividir os \textit{scanners} em dois tipos \cite{univhacker}: 

\begin{alineas}
\item \textbf{\textit{scanner} de portas TCP/IP abertas (ou \textit{portscanner})}: cada serviço de rede que estiver disponível em uma determinada máquina é uma porta de entrada em potencial. Existem um total de 128 mil porta, sendo 65536 portas para o protocolo TCP e 65536 portas para o protocolo UDP. O \textit{portscanner} verifica quais portas TCP/IP estão abertas com o objetivo de determinar quais serviços de rede TCP/IP disponíveis. Quase todas as técnicas de \textit{portscanning} valem-se de sinais (ou \textit{flags}), TCP, UDP ou ICMP, e a partir da análise desses sinais, os \textit{scanners} retiram informações sobre o sistema; \cite{univhacker}
\item \textbf{\textit{scanner} de vulnerabilidades conhecidas}: Um vez determinados os serviços que uma máquina disponibiliza na rede entra em cena o \textit{scanner} de vulnerabilidade. A ideia é checar, através de uma lista de falhas conhecidas, se o sistema está ou não executando um serviço com problemas \cite{univhacker}. 
\end{alineas}

Normalmente, essas ferramentas funcionam em três estágios \cite{avaliacao:tania}:

\begin{alineas}
 \item \textbf{Configuração}: aqui será definido o endereço IP do alvo ou a URL (Uniform Resource Locator) da aplicação Web e demais parâmetros, como, por exemplo, utilização de \textit{proxy}.
 \item \textbf{Rastreamento}: esse estágio, em \textit{scanners} de vulnerabilidade de aplicações web, o \textit{scanner} chama a primeira página web e então examina seu código procurando \textit{links}. Cada \textit{link} encontrado é registrado e este procedimento é repetido várias vezes até que \textit{links} e páginas não sejam mais encontrados.
 \item \textbf{Exploração}: vários testes são executados e as requisições e respostas são armazenadas e analisadas. Ao final, os resultados são exibidos ao usuário e podem ser salvos para uma análise posterior. 
\end{alineas}

Um bom \textit{scanner} de vulnerabilidade verifica itens como \cite{univhacker}: 

\begin{alineas}
\item \textbf{Erros comuns de configuração}: portas não utilizadas por nenhum serviço abertas;
\item \textbf{Configurações e senhas-padrões}: instalação de softwares deixando-os com as configurações de fabrica (com usuário e senha-padrão), por exemplo, usuário: admin, senha: admin. Outro problema é deixar serviços desnecessários ativados;
\item \textbf{Combinação óbvias de usuário e senha}: Usuário comuns tendem a colocar senhas fáceis de lembrar;
\item \textbf{Vulnerabilidades divulgadas}: Sempre que uma falha de segurança é divulgada há uma corrida dos desenvolvedores para saná-las. Em paralelo, existem \textit{hackers} que querem chegar aos sistemas vulneráveis antes de serem consertados.
\end{alineas}

Os \textit{scanners} de vulnerabilidades automatizados contêm, e atualizam regularmente, enormes bancos de dados de assinaturas de vulnerabilidades conhecidas para basicamente tudo o que está recebendo de informações em uma porta de rede, inclusive sistemas operacionais, serviços e aplicativos web \cite{hackers:stuart-joel}. 

\subsection{\textit{Exploit}} \label{sec:exploit}

Os atacantes exploram \textit{bugs} ou vulnerabilidades em programas para ter acesso ao sistema alvo. Infelizmente, existem milhares de \textit{bugs}, em 2013, por exemplo, foram reportados 103,000 \textit{bugs} no sistema operacional Ubuntu. Outro projetos de códigos fechados, possuem estatísticas similares \cite{aeg:thanassis}.

Diante das vulnerabilidades obtidas por um \textit{scanners} (\autoref{sec:scanners}), o passo seguinte seria usar um \textit{exploit} adequado. Os \textit{exploit} são pequenos utilitários usados para explorar vulnerabilidades específicas, podendo ser utilizados de forma \textit{"stand alone"}, ou seja, diretamente, ou podem ser incorporados à \textit{malwares} \cite{exploit:cassio}.

Para alguns \textit{exploits} funcionar, é necessário ter acesso ao \textit{shell} da máquina-alvo. Tal artificio pode ser conseguido através da execução de um cavalo de tróia (\autoref{sec:malwares}) pela vítima em seu sistema. O \textit{trojan} abre uma porta de comunicação e permite que o invasor tenha total controle sobre a máquina, dessa forma é possível executar \textit{exploits} para quebrar outros níveis de segurança \cite{univhacker}.

\subsection{Força Bruta} \label{sec:forçabruta}

Na segurança da informação, a autenticação é umas das áreas-chaves onde há a distinção de usuários autorizados de outros não-autorizados, tendo como principal vantagem ser de fácil implementação, não requerendo equipamentos, como leitores biométricos \cite{denise-lilian}.

Na literatura sobre segurança da informação, o fator humano é considerado o elo mais fraco. Muitos usuários, por conveniência, criam senhas de acesso fáceis e, em muitos casos, única para acessar diversos sistemas. Nesse ponto que \textit{hackers} iram atuar para ter acesso não-autorizado ao sistema. 

Existem três métodos mais usados por programas de quebra de senha: ataques de dicionário (ou lista de palavras), ataques híbridos e ataques de força-bruta. Nos ataques por dicionários, utilizam-se listas de palavras comuns: nomes próprios, marcas conhecidas, gírias, nomes de canções, entre outros, tais elementos conseguidos por engenharia social \cite{univhacker}. 

 Um ataque de força bruta consiste em gerar todas as permutações e combinações possíveis de senha, criptografar cada uma e comparar a senha gerada com a senha criptografada original até encontrar uma que seja igual \cite{md5crack2012}. 

 Esse tipo de ataque é facilmente detectável pois, além de gerar uma alta carga no servidor, gera uma grande quantidade de registros de logs. No entanto, caso a pessoa má intencionada, de alguma outra forma, tenha acesso ao arquivo de \textit{hash} ou a tabela de usuário de um banco de dados, com as senha criptografadas do sistema, ela pode usar o ataque de força bruta no arquivo em qualquer máquina, assim, impossibilitando a detecção do ataque.

 Muitos sistemas já possuem formas de contornar esse tipo de ataque, por exemplo, bloqueio de usuário ao errar a palavra-chave por uma certa quantidade de vezes. Outra forma, é colocar um tempo de expiração da senha, por exemplo, a senha deve ser trocada a cada trinta dias por uma diferente e nunca usada anteriormente, dessa maneira, inviabilizando a quebra de senha por força bruta. 

\subsection{Desfiguração de páginas} \label{sec:desfiguração}

A desfiguração de páginas, \textit{defacement} ou pichação ocorre quando o conteúdo da página \textit{web} de um site é alterado. O atacante (\textit{defacer}) consegue fazer alterações em páginas explorando vulnerabilidade nas aplicações \textit{web} que permite injeção de \textit{script} malicioso ou através de furto de senha de acesso à interface \textit{web} usadas para administração remota \cite{certs-ataques}.

Nos serviços \textit{web}, como por exemplo, apache2 existe um usuário especial, comumente chamado de \textit{www-data} ou algo semelhante. O usuário \textit{www-data}, na maioria das vezes, precisa apenas de permissões de leitura nos arquivos porém muitos gerentes de sistemas cujo a conscientização sobre segurança é insuficiente, designa permissões errôneas (escrita ou alteração), e caso haja um comprometimento, através, por exemplo, de injeção de código remoto PHP, do servidor, o atacante poderá alterar a maioria dos arquivos. A ocorrência amplamente disseminada de ataques de desfiguração de páginas Web é uma consequência direta dessa prática \cite{seguranca:william-lawrie}.

Esse tipo de ataque pode trazer sérias consequências à instituição, entre elas \cite{esr:tratamento}:

\begin{alineas}
\item \textbf{Constrangimento}: A instituição pode ter a imagem de confiabilidade afetada, em certos casos, refletir o descaso com que as informações críticas são tratadas; 
\item \textbf{Disseminação de inverdades}: Algumas alterações no \textit{website}, por exemplo, alterações de preços de produtos, podem resultar em consequências negativas; 
\item \textbf{Prejuízo de serviços}: Pode indisponibilizar serviços prestados pela instituição, por exemplo, em \textit{e-commence}.
\end{alineas}

Existem ferramentas que automatizam esse tipo de ataque, elas identificação aplicações web populares vulneráveis, de modo explorar falhas de segurança e alterar o conteúdo da página \cite{esr:tratamento}.

\begin{figure}[htb]
 \centering
 \caption{Estatísticas de \textit{defacement}}
 \includegraphics[scale=.7]{estatisticas-defacement.png}
 \legend{Fonte: \cite{zoneh}}
 \label{fig:estatistica-defacement}
\end{figure}

O site Zone-H mantém um arquivo de páginas alteradas. Os próprios \textit{hackers} submetem os \textit{websites} comprometidos no intuito de ter seus minutos fama. Nas submissões, os sites são espelhados para o Zone-H, então os moderadores verificam a veracidade do \textit{defacement}. Em 2013, foram identificados cerca de 70000 páginas comprometidas, desde então houve uma redução nesse número (\autoref{fig:estatistica-defacement}). Esse tipo de ataque é considerado passivo pois é gerado somente uma mensagem na tela \cite{angelo-xss}.

\subsection{Negação de Serviços} \label{sec:negação}
 
Um ataque de negação de serviço (\textit{Denial of Service} - DoS) tem como principal objetivo deixar um serviço (servidor \textit{web}, banco de dados) ou recurso (memória, processador)  indisponível, impossibilitando que usuário legítimos tenham acesso a esses recursos. Para tal, o atacante gera diversas requisições inúteis para o servidor, consumindo seus recursos até que o serviço não esteja mais disponível ou degradando a qualidade do serviço \cite{cryptsec}.

 Pode-se dividir os ataques de Dos em três categorias \cite{redes:kurose}:

\begin{alineas}
\item \textbf{Ataque de vulnerabilidade}: Envolve o envio de um série de mensagens a uma aplicação ou sistema operacional vulnerável, como consequência o serviço pode parar ou, no pior caso, o hospedeiro pode pifar;
\item \textbf{Inundação na largura de banda}: O atacante envia um grande quantidade de pacotes ao hospedeiro, fazendo com que o enlace de acesso do alvo fique indisponível, impedindo os pacotes legítimos de alcançarem o servidor;
\item \textbf{Inundação na conexão}: O atacante estabelece um grande número de conexões no hospedeiro-alvo fazendo-o deixar de aceitar conexões legítimas.
\end{alineas}

Esse tipo de ataque pode gerar grandes prejuízos financeiros para as empresas, principalmente \textit{e-commence}, pois enquanto o sistema está fora ou com uma resposta lenta, as transações financeiras são prejudicadas. Com isso, cria-se também, uma insatisfação pelo usuário do serviço prestado pela empresa.

Existe uma forma mais sofisticada de ataque de DoS chamada Negação de Serviço Distribuído (\textit{Distributed Denial of Services} - DDoS), enquanto o DoS básico as requisições partem de apenas uma fonte, no entanto, no DDoS o atacante tem acesso a um grande número de computadores (\textit{zombies}) explorando suas vulnerabilidades criando o que chamamos de \textit{botnet} (\autoref{fig:ddos}). Com isso, basta o atacante indicar as coordenadas de um ou mais alvos para o ataque \cite{zargarjoshitipper}. O DDoS são mais difíceis de detectar e de prevenir do que um ataque DoS de um único hospedeiro.

\begin{figure}[htb]
 \centering
 \caption{Ataque de Negação de Serviço Distribuído}
 \includegraphics[scale=.6]{ddos.png}
 \legend{Fonte: Autoria própria}}
 \label{fig:ddos}
\end{figure}

\subsection{\textit{Malwares}} \label{sec:malwares}

Os \textit{malwares}, também conhecidos como \textit{softwares} maliciosos, são um grande problema para sistemas de informação, sua existência ou execução tem consequências negativas ou involuntárias. Nessa seção será apresentado os \textit{malwares} mais popularmente conhecidos que são os vírus, \textit{worms}, \textit{trojans} e, devido sua repercussão, os \textit{ransomwares}.

É importante entender o funcionamento e o comportamento desses códigos maliciosos para, a partir daí, buscar soluções contra esse ataque. Existe dois tipos de análise: análise estática, requer uma verifica linha a linha do código malicioso, geralmente o código não está disponível e até mesmo se estiver, o autor do \textit{malware} muitas vezes ofusca o código, tornando esse tipo de análise difícil. Por outro lado, existe a análise dinâmica, o analista monitora a execução e o comportamento do \textit{malware}, esse tipo de análise é imune a ofuscação de código \cite{encycrypt}.

O Vírus é um programa que se propaga inserindo cópias de si mesmo e se tornando parte de outros programas e arquivos. Para dar continuidade ao processo de infecção, o vírus depende da execução do programa ou arquivo hospedeiro. O principal meio de propagação desse tipo de \textit{software} malicioso são as mídias removíveis, como, por exemplo, \textit{pen-drives} \cite{certs-malwares}.

O \textit{Worm} é um \textit{malware} que se propaga através de e-mails, sites ou \textit{software} baseados em rede, explorando as vulnerabilidades das aplicações. Uma das principais características desse tipo de \textit{software} é a propagação automática, ou seja, sem a intervenção do usuário \cite{detectingworm}. 

O \textit{Trojan} ou Cavalo de Troia são programas que precisam ser explicitamente executados para serem instalados no computador. Esse \textit{malware} se disfarça de um programa benigno, por exemplo, cartões virtuais animados, álbuns de fotos, jogos e protetores de tela que ao serem executados o \textit{trojan} é instalado sem o consentimento do usuário. No entanto, o atacante, após invadir um computador, pode instalar o \textit{trojan} alterando as funções já existentes de programas para executarem ações maliciosas \cite{certs-malwares}.

Por fim, temos os \textit{ransomwares}. O \textit{ransomware} é um \textit{malware} que criptografa os dados de um computador ou uma rede. A pessoa ou a organização responsável pelo ataque pede um resgate, geralmente pago em cripto moedas, como por exemplo, bitcoin, para manter sua anonimidade, fornecendo uma chave para descriptografar os arquivos mediante o pagamento \cite{ransomware:matt}.

A melhor medida contra esse tipo de \textit{malware}, uma vez que, não há garantias que o atacante irá fornecer a chave depois do pagamento, além de manter o sistema sempre atualizado, é ter uma politica de \textit{backup} regular. O armazenamento de arquivos importantes em outros tipos de mídias não conectadas regularmente ao sistema (removíveis) ou \textit{backup} baseados em nuvens \cite{ransomware:matt}.

\section{Ferramentas para Avaliação de Segurança} \label{sec:ferramentas}

Nessa seção será descrito as ferramentas auxiliares utilizadas para geração de ataques abordados na \autoref{sec:ataques-comuns} com objetivo de testar e validar as configurações das ferramentas de IDPS estudas. 

 \subsection{Nmap} \label{sec:nmap}

O Nmap é uma ferramente de código aberto utilizada para auditoria de segurança e descoberta de rede. A ferramenta é capaz de determinar quais \textit{hosts} estão disponíveis na rede, quais serviços cada \textit{host} está oferecendo, incluindo nome e versão da aplicação, o sistema operacional usado, dentre outras características.  

Muitos administradores de sistemas utilizam o Nmap para tarefas rotineiras como, criação de inventário de rede, gerenciamento de serviços, visto que é de suma importância manter os mesmos atualizados e monitoramento de \textit{host}.

 Diversos parâmetros podem ser utilizados com o Nmap, possibilitando realizar varreduras das mais variadas maneiras, dependendo do tipo desejado. A lista completa de opções podem ser consultadas na documentação oficial que vem junto da ferramenta ou no site do projeto \cite{nmap}. 

 Na execução do Nmap, o que não for opção ou argumento da opção é considerado especificação do \textit{host} alvo. O alvo pode ser um ou vários, usando uma notação de intervalo por hífen ou uma lista separada por vírgula. Os \textit{hosts} alvos também podem ser definidos em arquivos.

O resultado do Nmap é uma tabela de portas e seus estados (\autoref{fig:nmap-exemplo}). As portas podem assumir quatro estados, temos \cite{nmap}: 
\begin{alineas}
\item \textbf{\textit{open}}: significa que existe alguma aplicação escutando conexões; 
\item \textbf{\textit{filtered}}: há um obstáculo na rede, podendo ser algum \textit{firewall}, que impossibilita que o Nmap determine se a porta está aberta ou fechada; 
\item \textbf{\textit{closed}}: não possui aplicação escutando na porta; 
\item \textbf{\textit{unfilterd}}: a porta responde requisição porém o Nmap não consegue determinar se estão fechadas ou abertas.
\end{alineas}

 \begin{figure}[htb]
  \centering
  \caption{Exemplo de saída do Nmap}
  \includegraphics[scale=.6]{nmap.png}
  \legend{Autoria própria}
  \label{fig:nmap-exemplo}
 \end{figure}

\subsection{Metasploit Framework} \label{sec:metasploit}

O Metasploit é um \textit{framework} de código aberto cujo principio básico é desenvolver e executar \textit{exploit} contra alvos remotos e fornecer uma lista de vulnerabilidades existentes no alvo. É uma ferramenta que combina diversos \textit{exploits} e \textit{payloads} dentro de um local, ideal para levantamento de segurança de serviços e testes de penetração \cite{metasploit:yash}.  

O Metasploit possui uma biblioteca divida em três partes:

\begin{alineas}
\item \textbf{Rex}: É a biblioteca fundamental, a maioria das tarefas executadas pelo \textit{framework} usarão essa biblioteca; 
\item \textbf{MSF Core}: É o \textit{framework} em si, possui, por exemplo, gerenciador de módulos e a base de dados; 
\item \textbf{MSF Base}: Guarda os módulos, sejam eles, \textit{exploit}, \textit{encoders} (ferramentas usadas para desenvolver o \textit{payloads}) e os \textit{payloads}. Além disso, são guardadas informações de configuração e sessões criadas pelos \textit{exploits}. 
\end{alineas}

A Interface permite que o usuário interaja com o \textit{framework}. Nele há o \textit{msfconsole} uma interface de linha de comando interativa, o \textit{msfcli} interface de linha de comando não-interativa, e o \textit{msfweb} interface baseada em \textit{web} \cite{metasploittoolkit}. Por fim, temos o Armitage, que é uma interface gráfica baseada em Java desenvolvido por Raphael Mudge.

A arquitetura é mostrada com mais detalhes na \autoref{fig:metasploit-arquitetura}. 

Os módulos são divididos da seguinte maneira: 

\begin{alineas}
\item \textbf{Payload}: são código executados no alvo remotamente; 
\item \textbf{Exploit}: explora \textit{bugs} ou vulnerabilidade existente em aplicações do alvo; 
\item \textbf{Módulos Auxiliares}: usado para escanear as vulnerabilidades e executar várias tarefas; 
\item \textbf{Encoder}: codifica o \textit{payload} para evitar qualquer tipo de detecção por antivírus.
\end{alineas}

 \begin{figure}[!htb]
  \centering
  \caption{Arquitetura do Metasploit}
  \includegraphics[scale=.6]{metasploit_arquitetura.png}
  \legend{Fonte: Autoria própria}
  \label{fig:metasploit-arquitetura}
 \end{figure}

 \subsection{Pytbull} \label{sec:pytbull}

 O Pytbull é um \textit{framework} para teste de IDPS, capaz de determinar a capacidade de detecção e bloqueio do mesmo, além de fazer uma comparação entre diversas soluções e verifica as configurações \cite{pytbull}. O \textit{framework} Pytbull possui cerca de 300 testes agrupados em 11 módulos, temos:

 \begin{alineas}
  \item \textbf{badTraffic}: pacotes não compatíveis com a RFC são enviados para o servidor para testar como os pacotes são processados; 
  \item \textbf{bruteForce}: testa a capacidade do IDPS de rastrear ataques de força bruta;
  \item \textbf{clientSideAttacks}: usa um \textit{shell} reverso para fornecer ao servidor instruções para baixar arquivos maliciosos; 
  \item \textbf{denialOfService}: testa a capacidade do IDPS de proteger contra tentativas de DoS; 
  \item \textbf{evasionTechniques}: testa a capacidade do IDPS de detectar técnicas de evasão; 
  \item \textbf{fragmentedPackets}: várias cargas úteis fragmentadas são enviadas ao servidor para testar sua capacidade de recomposição e detectar os ataques; 
  \item \textbf{ipReputation}: testa a capacidade do servidor detectar tráfego de servidores com reputação baixa;
  \item \textbf{normalUsage}: cargas úteis que correspondem a uso normal; 
  \item \textbf{pcapReplay}: permite reproduzir arquivos pcap; 
  \item \textbf{shellCodes}: envia \textit{shellcodes} para o servidor na porta 21/ftp testando a capacidade de detectar e/ou bloquear o mesmo; 
  \item \textbf{testRules}, testa a base de assinaturas configuradas no servidor IDPS.
 \end{alineas}

Existem basicamente 5 tipos de testes \cite{pytbull}: 
\begin{alineas}
\item \textbf{\textit{socket}}: Abre um \textit{socket} em uma porta e envia o \textit{payload} para o alvo remoto na porta especificada; 
\item \textbf{\textit{command}}: Envia um comando para alvo remoto com a função python subprocess.call(); 
\item \textbf{\textit{scapy}}: Envia cargas úteis especificas baseadas na sintaxe de Scapy;
\item \textbf{\textit{client side attacks}}: Usa um \textit{shell} reverso no alvo remoto e envia comandos para serem processados no servidor; 
\item \textbf{\textit{pcap replay}}: Permite reproduzir tráfego com base em arquivos de pcap.
\end{alineas}

 \begin{figure}[htb]
  \centering
  \caption{Arquitetura do \textit{framework} Pytbull}
  \includegraphics[scale=.4]{arquitetura_pytbull.png}
  \legend{Fonte: \cite{pytbull}}
  \label{fig:pytbull}
 \end{figure}

 \section{Conclusão}

 Este capítulo apresentou definições sobre segurança da informação, mostrando os elementos envolvidos. Definindo os tipos de ataques existentes e suas categorias e os possíveis impactos. Mostrou-se um cenário geral de uma rede de computador e seus componentes assim como suas definições e os ataques comuns envolvendo essas redes. Ao final do capítulo, mostrou-se as ferramentas auxiliares usadas para simular ataques com objetivo de analisar o comportamento dos IDPS.
